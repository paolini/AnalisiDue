\chapter{funzioni di più variabili}

Norme equivalenti sullo spazio delle matrici. Lipschitzianità delle funzioni $C^1$.
Le funzioni continue mandano compatti in compatti.
Se $A,B$ sono compatti, se $f\colon A\to B$ e continua e bigettiva allora 
$f^{-1}$ è continua (cioè $f$ è un omeomorfismo).

%Definire $C^1(A)$ come lo spazio delle funzioni su $A$ che possono essere 
%estese di classe $C^1$ ad un aperto contenente $A$.

\begin{theorem}[Lagrange vettoriale]
  \label{th:lagrange}
Sia $\Omega\subset \RR^n$ un aperto convesso, 
sia $\vec f\colon \Omega\to \RR^m$ una funzione differenziabile.
Allora dati $\vec x,\vec y \in \Omega$, $\vec x\neq \vec y$, 
esiste $\vec z \in [\vec x,\vec y]$ (il segmento che congiunge $\vec x$ e $\vec y$)
tale che 
\[
 \abs{\vec f(\vec y)- \vec f(\vec x)}^2
 =  D\vec f(\vec z)[\vec y-\vec x]\cdot (\vec f(\vec y)-\vec f(\vec x)).
\]
In particolare si ha 
\[
\abs{\vec f(\vec y)- \vec f(\vec x)}
\le \max_{\vec z\in [\vec x,\vec y]} \abs{D\vec f(\vec z)} \cdot \abs{\vec y-\vec x}.
\]
\end{theorem}

\begin{exercise}[punto di frontiera]
  \label{ex:punto-di-frontiera}
Sia $\gamma\colon [0,1] \to X$ una curva continua in uno 
spazio topologico $X$ e sia $A\subset X$ tale che 
$\gamma(0)\in A$ e $\gamma(1)\notin A$.
Dimostrare che esiste $t \in [0,1]$ tale che $\gamma(t) \in \partial A$.
\end{exercise}

