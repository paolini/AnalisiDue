\section{convergenza uniforme}

\begin{example}[Weiestrass]
Posto $a=\frac 1 8$ e $b=512$, 
consideriamo la funzione $W\colon \RR \to \RR$, definita da
\[
   W(x) = \sum_{k=0}^{+\infty} a^k \cos(b^k \pi x).
\]
La funzione $W$ 
è continua in ogni punto di $\RR$ 
ma non è derivabile in nessun punto di $\RR$.
\end{example}
\begin{proof}
Posto $f_k(x) = a^k \cos(b^k \pi x)$,
si ha $\Abs{f_k} \leq a^k$ e quindi, essendo $\sum a^k$ convergente, 
la serie di funzioni $W = \sum f_k$
converge totalmente su $\RR$ e dunque converge uniformemente.
Essendo ogni $f_k$ continua, anche $W$ è continua.

E' molto più delicato dimostrare che $W$ non è derivabile in nessun punto di $\RR$.
L'idea è che la funzione $f_k$ ha una derivata 
che oscilla con ampiezza $\pi a^k b^k$, mentre il suo modulo, come già visto,
ha ampiezza $a^k$. 
Se prendiamo due punti che hanno una distanza dell'ordine di $b^m$,
la funzione $f_m$ è quella che contribuisce maggiormente alla differenza 
del rapporto incrementale, in quanto per $k<m$ le funzioni oscillano relativamente 
poco, mentre per $k>m$ le funzioni hanno ampiezza molto piccola.
Sviluppiamo questa idea.

Fissiamo $x\in \RR$ qualunque. 
Il nostro obiettivo è trovare una successione $\delta_n\to 0$ 
tale che si abbia
\begin{equation}\label{eq:30234354}
  \abs{\frac{W(x+\delta_n) - W(x)}{\delta_n}} \to +\infty.
\end{equation}
Questo implica che $W$ non è derivabile in $x$.
Visto che il coseno ha periodo $2\pi$ siamo certi che nell'intervallo 
$[4\pi,6\pi]$ assume tutti i valori tra $-1$ e $1$. 
Dunque esiste $\theta_n \in [4\pi,6\pi]$ tale che 
\[
   \abs{\cos (b^n \pi x + \theta_n) - \cos(b^n\pi x)} = 1.
\]
Poniamo $\delta_n = \frac{\theta_n}{b^n \pi}$ cosicché
\[
  \abs{\cos (b^n \pi (x+\delta_n))- \cos(b^n \pi x)} = 1.
\]
Abbiamo quindi:
\begin{align*}
    \abs{\frac{W(x+\delta_n) - W(x)}{\delta_n}} 
    &= \frac{\pi b^n}{\theta_n}\cdot \abs{\sum_{k=0}^{+\infty}a^k\Enclose{\cos(b^k\pi x + b^{k-n}\theta_n) - \cos(b^k \pi x)}}  \\
    &\ge \frac{\pi}{6}b^n a^n\cdot 1 - \frac{\pi b^n}{\theta_n}\Enclose{\sum_{k=0}^{n-1} a^k b^{k-n} \theta_n + \sum_{k=n+1}^{+\infty} 2a^k} \\
    &\ge \frac{a^n b^n}{6} - \frac{b^n}{4}\Enclose{\frac{6\pi}{b^n}\sum_{k=0}^{n-1} (ab)^k + 2 a^{n+1} \frac{1}{1-a}} \\
    &\ge \frac{a^n b^n}{6} -\frac {3\pi} 2\cdot  \frac{(ab)^n-1}{ab-1} - \frac{a^n b^n}{2}\cdot \frac{a}{a-1}\\
    &\ge (ab)^n \Enclose{\frac 1 6 - 5\cdot \frac{1}{ab-1} - \frac 1 2 \cdot \frac{1}{1-a}} \\
    &= (ab)^n \frac{(ab-1)(1-a)-30(1-a)-3a(ab-1)}{6(ab-1)(1-a)}.
\end{align*}
Le scelte di $a$ e $b$ garantiscono che
si abbia $(ab-1)>0$, $1-a>0$ e
\begin{align*}
(ab-1)(1-a)-30(1-a)-3a(ab-1)
&= ab(1-4a) - 31 + 34a > 0.
\end{align*}
Si ottiene dunque \eqref{eq:30234354} come voluto.
\end{proof}