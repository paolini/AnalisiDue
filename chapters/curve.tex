\begin{theorem}[riparametrizzazione]
Sia $\vec u \colon [a,b] \to X$ una curva in uno spazio metrico $X$ 
e sia $\phi \colon [A,B] \to [a,b]$ una funzione bigettiva, monotona.

Allora posto $\vec v = \vec u \circ \phi$ si ha
\[
  \ell(\vec u, a, b) = \ell(\vec v, A, B).
\]

Osservazione: $\phi$ bigettiva e monotona implica che $\phi^{-1}$ è anch'essa
bigettiva e monotona e che $\phi$ e $\phi^{-1}$ sono continue.
\end{theorem}
\begin{proof}
La dimostrazione è ovvia, perché $\phi$ induce una corrispondenza biunivoca 
tra le suddivisioni di $[a,b]$ e quelle di $[A,B]$ che conserva le
lunghezze delle curve spezzate.
\end{proof}

\begin{definition}[equivalenza di curve]
Due curve $\vec u \colon [a,b] \to X$ e $\vec v \colon [A,B] \to X$ 
si dicono essere \emph{equivalenti} se esiste una funzione
$\phi \colon [A,B] \to [a,b]$ bigettiva e crescente tale che 
\[
    \vec u = \vec v \circ \phi.
\]

Osservazione: se una curva ha dei tratti su cui è costante, in base 
a questa definizione tutte le curve equivalenti avranno anch'esse dei corrispondenti 
tratti costanti. Si potrebbe decidere che per valutare l'equivalenza 
tra due curve si debbano prima eliminare i tratti costanti.
\end{definition}

\begin{definition}[lunghezza d'arco]
Sia $\vec u \colon [a,b] \to X$ una curva rettificabile in uno spazio metrico $X$.
Sia $L = \ell(\vec u, a, b)$ la lunghezza della curva.
La \emph{lunghezza d'arco} di $\vec u$ è la funzione $s\colon [a,b] \to [0,L]$ 
definita da
\[
  s(t) = \ell(\vec u, a, t).
\]
\end{definition}

Chiaramente la funzione $s$ è crescente. 
Se $\vec u$ non ha intervalli su cui è costante allora $s$ è strettamente crescente,
altrimenti $s$ è costante sugli intervalli in cui $\vec u$ è costante.
Si nota infatti che $\ell(\vec u, t_1, t_2) = 0$ se e solo se $\vec u$ è costante su $[t_1,t_2]$.

\begin{theorem}[continuità della lunghezza d'arco]
Sia $\vec u \colon [a,b] \to X$ una curva continua rettificabile in uno spazio metrico $X$.
Allora la funzione lunghezza d'arco $s\colon [a,b] \to [0,L]$ è continua.
\end{theorem}
\begin{proof}
Fissato $t_0 \in [a,b]$ dobbiamo mostrare che $s$ è continua in $t_0$.
Mostreremo che $s(t)\to s(t_0)$ per $t\to t_0^+$, il caso $t\to t_0^-$ si tratta 
in maniera analoga.

Visto che $s$ è una funzione crescente le eventuali discontinuità sono 
discontinuità \emph{a salto}. 
Se non fosse continua a destra nel punto $t_0$ si avrebbe
\[
    \lambda = \inf_{t>t_0} s(t) - s(t_0) > 0.
\]
Visto che $\vec u$ è continua, esiste $\delta>0$ tale che per ogni $t \in [t_0, t_0+\delta]$
si ha $d(\vec u(t), \vec u(t_0)) < \frac \lambda 2$.
Visto che $\vec u$ è rettificabile, esistono $t_0<t_1<\dots < t_N \le t_0+\delta$
tali che 
\[
  \ell(\vec u, t_0, t_0+\delta) \le \sum_{k=1}^N d(\vec u(t_{k-1}), \vec u(t_k)) + \frac \lambda 2.
\]
Ma 
\[
    \ell(\vec u, t_0, t_0+\delta) = \sum_{k=1}^N \ell(\vec u, t_{k-1}, t_k)
\]
dunque 
\[
  d(\vec u(t_0), \vec u(t_1)) \ge \ell (\vec u, t_0,t_1) - \frac \lambda 2
  = s(t_1)-s(t_0) - \frac \lambda 2 
  \ge \lambda - \frac \lambda 2 = \frac \lambda 2.
\]
Questo contraddice la scelta di $\delta$ e conclude la dimostrazione.
\end{proof}

\begin{theorem}[semicontinuità della lunghezza]
Sia $u_n\colon [a,b] \to X$ una successione di curve continue
in uno spazio metrico $X$ che converge uniformemente
ad una curva continua $\vec u \colon [a,b] \to X$.
Allora 
\[
  \ell(\vec u, a, b) \le \liminf_{n\to \infty} \ell(\vec u_n, a, b).
\]
\end{theorem}
\begin{proof}
Per ogni $\eps>0$ esiste $n$ tale che $d_\infty(\vec u_n, \vec u) < \eps$.
Se $P={t_0,\dots, t_N}$ è una suddivisione di $[a,b]$
allora
\[
    \ell(u,P) \defeq \sum_{k=1}^N d(u(t_{k-1}), u(t_k))
     \le \sum_{k=1}^N \bigl( d(u_n(t_{k-1}), u_n(t_k)) + 2\eps \bigr)
        = \ell(u_n,P) + 2 N \eps.
\]
Dunque 
\[
 \ell(u,a,b) \le \ell(u,P) \le \ell(u_n,a,b) + 2N\eps 
 \le \liminf_{k\to \infty} \ell(u_k,a,b) + 2N\eps.
\]
??? rivedere ???
\end{proof}

\begin{theorem}[Ascoli-Arzelà]
Siano $\vec u_n \colon [a,b] \to \RR$ funzioni tali che 
    \begin{enumerate}
    \item le funzioni $\vec u_n$ sono uniformemente limitate, cioè esiste
    $M>0$ tale che per ogni $n$ e per ogni $t\in [a,b]$ si ha 
    $\abs{\vec u_n(t)} \le M$;
    \item le funzioni $\vec u_n$ sono equicontinue, cioè per ogni $\eps>0$ esiste 
    $\delta>0$ tale che per ogni $n$ e per ogni $t_1,t_2 \in [a,b]$ si ha 
    \[
    \abs{t_1-t_2}<\delta \implies 
    \abs{\vec u_n(t_1) - \vec u_n(t_2)} < \eps.
    \]
    \end{enumerate} 
Allora esiste una sottosuccessione $\vec u_{n_k}$ che converge uniformemente
ad una funzione continua $\vec u \colon [a,b] \to \RR$.
\end{theorem}   

\begin{proof}
\emph{Claim: per ogni $\eps>0$ esiste $n_k\in \NN$ successione strettamente 
crescente di indici, 
tale che $\Abs{\vec u_{n_k}(t) - \vec u_{n_j}(t)}_\infty < \eps$ per ogni $k,j\in \NN$.}
Dato $\delta>0$ scegliamo $N\in \NN$ tale che $N \delta < b-a$ e 
consideriamo $a=x_0 < x_1 < x_2 < \dots < x_N=b$ con $x_{i+1}-x_i < \delta$.
La successione $(u_n(x_1), u_n(x_2), \dots, u_n(x_N))\in \RR^N$ 
è limitata dunque, per il teorema di Bolzano-Weierstrass,
ammette una estratta $(u_{n_k}(x_1),\dots,u_{n_k}(x_N))$ convergente in $\RR^N$.
Tale successione è quindi di Cauchy in $\RR^N$. 
Dunque 
dato $\eps>0$ se $k,j$ sono sufficientemente grandi e per ogni $n=1,\dots,N$ si ha
\[
  \abs{u_{n_k}(x_n) - u_{n_j}(x_n)} < \eps.
\]
A meno di eliminare i primi indici della successione possiamo supporre che
questa disuguaglianza valga per ogni $k,j\in \NN$.

Ora, per l'equicontinuità, per ogni $t\in [a,b]$ esiste $x_n$ tale che
$\abs{t - x_n} < \delta$ e dunque
\begin{align*}
  \abs{u_{n_k}(t) - u_{n_j}(t)} 
  &\le \abs{u_{n_k}(t) - u_{n_k}(x_n)} + \abs{u_{n_k}(x_n) - u_{n_j}(x_n)} + \abs{u_{n_j}(x_n) - u_{n_j}(t)} \\
  &\le \eps + \eps + \eps = 3 \eps.
\end{align*}
Questo conclude la dimostrazione del claim.

\emph{Conclusione.}
Trovo $n_k^1$ tale che $\Abs{u_{n_k^1} - u_{n_j^1}}_\infty < 1$ per ogni $k,j$.
Poi trovo $n_k^2$ sottosuccessione estratta di $n_k^1$ tale che
$\Abs{u_{n_k^2} - u_{n_j^2}}_\infty < \frac 1 2$ e, 
procedendo in questo modo per induzione,
\[
  \Abs{u_{n_k^m} - u_{n_j^m}}_\infty < \frac 1 m \qquad \forall k,j \in \NN.
\]
La successione diagonale $u_{n_k^k}$ ha la proprietà che
\[
  \Abs{u_{n_k^k} - u_{n_j^j}} \le \frac 1 k \qquad \forall j\ge k.
\]
Quindi $u_{n_k^k}$ è di Cauchy in $C^0([a,b],\RR)$ rispetto alla distanza 
uniforme e, per la completezza di tale spazio, converge uniformemente 
ad una funzione continua $u \colon [a,b] \to \RR$.
\end{proof}

La stessa dimostrazione permette di dimostrare il seguente risultato 
più astratto.

\begin{abstract}[Ascoli-Arzelà più astratto]
Sia $X$ uno spazio metrico compatto, $Y$ uno spazio metrico. 
Sia $u_n\colon X \to Y$ una successione di funzioni tali che 
\begin{enumerate}
    \item per ogni $x\in X$ la successione $u_n(x)$ è relativamente compatta 
    in $Y$ (cioè ammette una estratta convergente);
    \item le funzioni $u_n$ sono equicontinue.
\end{enumerate}
Allora esiste una sottosuccessione $u_{n_k}$ che converge uniformemente
ad una funzione continua $u\colon X \to Y$.
\end{abstract}

Come applicazione del teorema di Ascoli-Arzelà
proponiamo il seguente importante risultato.

\begin{theorem}[esistenza delle geodetiche]
    Sia $X$ uno spazio metrico completo, proprio 
    (cioè ogni chiuso limitato è compatto), 
    e sia $G(x,y)$ l'insieme di tutte le curve continue di lunghezza finita 
    $u\colon [0,1] \to X$
    tali che $u(0) = x$ e $u(1) = y$.
    Allora, se $G(x,y)$ non è vuoto, il funzionale \emph{lunghezza}
    $\ell\colon G \to \RR$ ammette un minimo.
\end{thorem}
\begin{proof}
    Sia $L = \inf\ell(G)$ e sia 
    $u_n\in G(x,y)$ una successione minimizzante,
    cioè tale che $\ell(u_n) \to L$.
    Posto $L_n = \ell(u_n)$ posso considerare la riparametrizzazione 
    di $u_n$ per lunghezza d'arco. 
    Questo mi dà una curva $\tilde u_n \colon [0,L_n] \to X$
    che è $1$-lipschitziana in quanto 
    $\ell(\tilde u_n, s,t) = t-s$.
    Le curve $v_n \colon [0,1] \to X$ definite da
    \[
        v_n(t) = \tilde u_n (L_n t)
    \]
    saranno invece $L_n$-lipschitziane. 
    Poiché $L_n\to L<+\infty$ esiste $M = \max L_n < +\infty$ 
    e le curve $v_n$ sono tutte $M$-lipschitziane e dunque 
    sono equicontinue. 
    Tutte queste curve sono contenute nella chiusura della 
    palla di centro $x$ e raggio $M$, che è compatta
    perché abbiamo supposto $X$ proprio.
    Dunque, per il teorema di Ascoli-Arzelà, esiste una sottosuccessione 
    $v_{n_k}$ che converge uniformemente ad una curva continua
    $v\colon [0,1] \to X$.
    La curva $v$ è $M$-lipschitziana in quanto limite uniforme di 
    $M$-lipschitziane, dunque è di lunghezza finita e 
    appartiene a $G(x,y)$.
    Infine, per la semicontinuità inferiore della lunghezza
    \[
        \ell(v) \le \liminf_{k\to \infty} \ell(v_{n_k}) = L,
    \]
    dunque $v$ è una curva di lunghezza minima.
\end{proof}

