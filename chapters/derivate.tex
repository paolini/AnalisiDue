\chapter{funzioni di più variabili}

\section{metrica e topologia di $\RR^n$}

L'insieme $\RR^n$ è l'insieme delle $n$-uple di 
numeri reali: se $\vec x \in \RR^n$ si ha $\vec x=(x_1,\dots,x_n)$ 
con $x_k\in \RR$ per $k=1,\dots,n$.

$\RR^n$ è uno spazio vettoriale.

Ripasso di algebra lineare. Applicazioni lineari. 
Rappresentazione dei sottospazi lineari e affini in forma implicita (kernel)
in forma parametrica (immagine), in forma di grafico.

$\RR^n$ è dotato di prodotto scalare. 
Isomorfismo tra spazio e duale indotto dal prodotto scalare.
Il prodotto scalare induce una norma.
La norma induce una distanza.

Cauchy-Schwarz. Pitagora. 

La distanza rende $\RR^n$ uno spazio metrico 
completo.

Compattezza vera. Totale limitatezza. Compattezza sequenziale.

\begin{definition}[successione di Cauchy]
\end{definition}

\begin{definition}[completezza]
\end{definition}

Tramite la distanza di definiscono le palle. 
Tramite le palle gli intorni.
Gli intorni di uno spazio metrico inducono una 
topologia.

\section{derivate}

Derivate parziali. 
Differenziablità.
Derivate direzionali.
Gradiente. 
Matrice delle derivate.
Teorema del differenziale.

Derivate seconde.
Teorema di Schwarz.

Norme equivalenti sullo spazio delle matrici. Lipschitzianità delle funzioni $C^1$.
Le funzioni continue mandano compatti in compatti.
Se $A,B$ sono compatti, se $f\colon A\to B$ e continua e bigettiva allora 
$f^{-1}$ è continua (cioè $f$ è un omeomorfismo).

Definire $C^1(A)$ come lo spazio delle funzioni su $A$ che possono essere 
estese di classe $C^1$ ad un aperto contenente $A$.

