\chapter{integrali multipli}

\section{integrale di Riemann}
Se $x,y\in \RR$ definiamo:
\[
    x \vee y = \max\ENCLOSE{x,y},
    \quad
    x \wedge y = \min\ENCLOSE{x,y},
    \quad
    x^+ = x \vee 0,
    \quad
    x^- = (-x)^+ = -(x\wedge 0).
\]
Si ha
\[
    x^+,x^-\ge 0,
    \quad
    x = x^+ - x^-,
    \quad 
    \abs{x} = x^+ + x^-,
    \quad
    x^+ = \frac{\abs{x}+x}{2},
    \quad
    x^- = \frac{\abs{x}-x}{2}.
\]

Sia $A\subset \RR^n$ e $f\colon A \to \RR$. 
Definiamo il \emph{supporto} di $f$ come 
\[
 \spt f 
    = \overline{\ENCLOSE{\vec x \in A \colon f(\vec x) \neq 0}}
    = \overline{f^{-1}(\RR\setminus\ENCLOSE{0})}.
\]

Dato $A\subset \RR^n$ definiamo la \emph{funzione caratteristica}
di $A$ come $1_A\colon \RR^n \to \RR$,
\[
  1_A(\vec x) = \begin{cases}
    1 & \text{se $\vec x \in A$,}\\
    0 & \text{altrimenti.}
  \end{cases}
\]

Per $\vec k \in \ZZ^d$ e $N\in \NN$ definiamo 
il cubetto 
\[
  Q^N_{\vec k} = \frac{\vec k + [0,1]^d}{2^N}.
\]
Si tratta di un cubetto di lato $2^{-N}$ con gli spigoli paralleli 
agli assi cartesiani e con un vertice nel punto $2^{-N} \vec k$.
Il volume di $Q^N_{\vec k}$ sarà quindi $2^{-nN}$.
Useremo questi cubetti per misurare gli insiemi.

\begin{definition}[integrale di Riemann]
  \label{def:integrale_riemann}
Sia $f\colon \RR^n \to \RR$ una funzione limitata, 
con supporto limitato.
Definiamo le somme superiori e inferiori sulla $N$-esima suddivisione in cubi:
\begin{align*}
    S^N(f) &= 2^{-nN}\sum_{\vec k\in \ZZ^n} \sup f(Q^N_{\vec k}), \\
    S_N(f) &= 2^{-nN}\sum_{\vec k\in \ZZ^n} \inf f(Q^N_{\vec k}).
\end{align*}
Visto che il supporto di $f$ è limitato, queste somme sono in realtà 
somme finite. Visto che $f$ è limitata, ogni addendo è finito e dunque 
$S^N(f), S_N(f) \in \RR$ per ogni $N\in \NN$.
Osservando che $N\mapsto S^N(f)$ è decrescente 
mentre $N\mapsto S_N(f)$ è crescente, definiamo
l'integrale superiore e inferiore:
\begin{align*}
    S^*(f) &= \lim_{N\to +\infty} S^N(f) \in \RR, \\
    S_*(f) &= \lim_{N\to +\infty} S_N(f) \in \RR.
\end{align*}
Se $S^*(f)=S_*(f)$ diremo che $f$ è Riemann integrabile (o R-integrabile)
e scriveremo
\[
  \int f = S^*(f) = S_*(f).
\]
\end{definition}

\begin{theorem}[monotonia dell'integrale]
Siano $f,g\colon \RR^n\to\RR$ due funzioni limitate con supporto limitato.
Se $f$ e $g$ sono $R$-integrabili si ha:
\[
  f \le g \implies \int_A f \le \int_A g.
\]
\end{theorem}
%
\begin{proof}
Se $f\le g$, per ogni $N\in \NN$ e ogni $\vec k \in \ZZ^n$ si ha 
\[
    \sup f(Q^N_{\vec k}) \le \sup g(Q^N_{\vec k}) 
\]
da cui 
\[
    S^N(f) \le S^N(g)
\]
e passando al limite per $N\to +\infty$,
\[
  S^*(f) \le S^*(g)
\]
che è la tesi se $f$ e $g$ sono R-integrabili.
\end{proof}

\begin{theorem}[linearità dell'integrale]
Siano $f,g\colon \RR^n \to \RR$ funzioni limitate con supporto limitato.
Se $f,g$ sono R-integrabili allora $f+g$ è R-integrabile e si ha 
\[
    \int (f+g) = \int f + \int g.
\]
Se $f$ è R-integrabile e $\lambda\in \RR$ anche $\lambda f$ è R-integrabile 
e si ha 
\[
    \int \lambda f = \lambda \int f.
\]
\end{theorem}
\begin{proof}
Per ogni $\vec k\in\ZZ^n$ e ogni $N\in \NN$
si ha
\begin{align*}
  \sup\, (f+g)(Q^N_{\vec k}) &\le \sup f(Q^N_{\vec k}) + \sup g(Q^N_{\vec k})\\
  \inf\, (f+g)(Q^N_{\vec k}) &\ge \inf f(Q^N_{\vec k}) + \inf g(Q^N_{\vec k}).
\end{align*}
Dunque si ottiene
\[
    S^N(f+g) \le S^N(f) + S^N(g),  \qquad
    S_N(f+g) \ge S_N(f) + S_N(g), 
\]
e passando al limite per $N\to +\infty$ 
\[
   S_*(f) + S_*(g) \le S_*(f+g) \le S^*(f+g) \le S^*(f) + S^*(g).
\]
Se $f$ e $g$ sono R-integrabili si ottiene dunque che anche $f+g$ 
è R-integrabile e inoltre $\int (f+g) = \int f + \int g$.

Se $\lambda \ge 0$ si ha 
\[
   S^N(\lambda f) = \lambda S^N(f), \qquad S_N(\lambda f) = \lambda S_N(f)
\]
e quindi se $f$ è R-integrabile si conclude 
che $\lambda f$ è R-integrabile e $\int \lambda f = \lambda \int f$.

Se $\lambda = -1$ si osserva che per ogni $N\in \NN$, $\vec k\in \ZZ^n$, 
si ha 
\[
   \sup (-f)(Q^N_{\vec k}) = -\inf f(Q^N_{\vec k}),
   \qquad 
   \inf (-f)(Q^N_{\vec k}) = -\sup f(Q^N_{\vec k}),
\]
da cui 
\[
    S^N(-f) = -S_N(f), \qquad S_N(-f) = -S^N(f)
\]
e quindi, per $N\to +\infty$ si ottiene 
\[
  S^*(-f) = -S_*(f), \qquad S_*(-f) = - S^*(f).
\]
Se $f$ è R-integrabile deduciamo che anche $-f$ lo è e si ha 
$\int (-f) = -\int f$.

Combinando i casi precedenti si ottiene il caso generale $\lambda \in \RR$.
\end{proof}

\begin{theorem}[proprietà di reticolo]
Sia $f\colon \RR^n \to \RR$ una funzione limitata con supporto limitato.
Se $f$ è R-integrabile allora anche $\abs{f}$, $f^+$ e $f^-$ sono 
R-integrabili e si ha 
\[
  \abs{\int f} \le \int \abs f = \int f^+ + \int f^-, 
  \qquad 
 \int f = \int f^+ - \int f^-.
\]
\end{theorem}
\begin{proof}
Dalla proprietà generale 
\[
  \abs{\rule{0pt}{1.1em}\abs{f(\vec x)} - \abs{f(\vec y)}} \le \abs{f(\vec x) - f(\vec y)}
\]
si ottiene che per ogni $N\in \NN$, $\vec k \in \ZZ^n$, vale
\[
  \sup \abs{f}(Q^N_{\vec k}) - \inf \abs{f}(Q^N_{\vec k})
  \le 
  \sup f(Q^N_{\vec k}) - \inf f(Q^N_{\vec k})
\]
e quindi 
\[
   S^N(\abs f) - S_N(\abs f) \le S^N(f) - S_N(f)
\] 
da cui, passando al limite per $N\to +\infty$,
\[
  S^*(\abs f) - S_*(\abs f) \le S^*(f) - S_*(f)
\]
per cui se $f$ è R-integrabile anche $\abs{f}$ lo è.

Osserviamo ora che $f^+=\frac{\abs f + f}{2}$ e $f^-=\frac{\abs f - f}{2}$ 
e dunque per linearità se $f$ è R-integrabile avendo già dimostrato 
che anche $\abs f$ è R-integrabile, si ottiene che pure $f^+
$ e $f^-$ sono R-integrabili ed essendo $f=f^+-f^-$, $\abs f = f^+ + f^-$ si ottiene 
\[
  \int f = \int f^+ - \int f^-, \qquad 
  \int \abs f = \int f^+ + \int f^-.
\]
Per la convessità del valore assoluto si ottiene infine
\[
 \abs{\int f} = \abs{\int f^+ - \int f^-} 
 \le \abs {\int f^+} + \abs{\int f^-}
 = \int f^+ + \int f^- = \int \abs f.
\]
\end{proof}

\begin{definition}[integrale di Riemann generalizzato]
Se $f\colon \RR^n \to [0,+\infty]$ è una funzione (non negativa), per ogni $R>0$ 
possiamo definire il troncamento $f_R$ 
\[
  f_R = (f \cdot 1_{[-R,R]^n}) \vee R.
\]
La funzione $f_R$ è limitata ed ha supporto limitato.
Se $f_R$ è Riemann-integrabile per ogni $R>0$ diremo che $f$ è localmente R-integrabile 
e visto che $R\mapsto f_R(\vec x)$ è crescente possiamo definire
\[
 \int f = \lim_{R\to+\infty} \int f_R \in [0,+\infty].
\]
Se $\int f<+\infty$ diremo che $f$ è R-integrabile (avendo supposto $f\ge 0$)

Se $f\colon \RR^n\to \RR$ possiamo scrivere $f= f^+-f^-$ con $f^+,f^-$ funzioni non negative.
Se $f^+$ e $f^-$ sono localmente R-integrabili e se 
$\int f^+$ e $\int f^-$ non sono entrabi infiniti possiamo definire 
\[
  \int f = \int f^+ - \int f^- \in [-\infty,+\infty].
\]
Se $\int f$ è finito diremo che $f$ è $R$-integrabile.
\end{definition}

\begin{definition}
Dato $E\subset \RR^n$ diremo che $E$ è \emph{Peano-Jordan misurabile} 
(o PJ-misurabile) se la funzione caratteristica $1_E$ è localmente 
R-integrabile. In tal caso definiamo la misura di $E$ come 
\[
\abs{E} = \int 1_E \in [0,+\infty].
\] 

Se $A\subset \RR^n$ è PJ-misurabile e $f\colon A \to \RR$
è una funzione qualunque, possiamo considerare l'estensione 
$\tilde f\colon \RR^n \to \RR$ definita da 
\[
 \tilde f(\vec x) = \begin{cases}
    f(\vec x) & \text{se $x\in A$,}\\
    0 & \text{altrimenti.}
 \end{cases}
\]
Diremo che $f$ è localmente R-integrabile in $A$ se $\tilde f$ 
è localmente R-integrabile. 
Scriveremo:
\[
 \int_A f = \int \tilde f
\]
se $\int \tilde f$ è definito.
Se inoltre $\int_A f$ è finito diremo che $f$ è R-integrabile
su $A$.

Volendo rendere esplicita la variabile di integrazione, potremo 
scrivere 
\[
  \int_A f(x)\, dx = \int_A f.
\]
\end{definition}

\begin{remark}[caso $n=1$]
Nel caso $n=1$ se $f\colon [a,b]\to \RR$ è una funzione limitata,
si osserva che la suddivisione in cubetti di lato $2^{-N}$ induce una partizione 
su $[a,b]$ e la definizione di Riemann integrabilità data sopra coincide con la 
usuale Riemann integrabilità definita tramite suddivisioni arbitrarie.

Nel caso $f\colon(a,b)\to \RR$ con $(a,b)$ non limitato oppure $f$ non limitata,
la definizione data sopra corrisponde invece alla \emph{assoluta convergenza} 
dell'integrale improprio.
\end{remark}

\begin{theorem}[algebra dei misurabili]
Se $E,F\subset \RR^n$ sono PJ-misurabili allora anche $\bar E$, $\partial E$, 
$E\setminus F$, $E\cap F$ ed $E\cup F$ sono PJ-misurabili.
Inoltre si ha 
\[
    \abs{E\cup F} + \abs{E\cap F} = \abs{E} + \abs{F},
    \qquad
    \abs{\partial E} = 0.
\]
\end{theorem}
\begin{proof}
Per la misurabilità di $E\setminus F$ e di $E\cap F$, basta osservare che 
\begin{align*}
  1_{E\setminus F} &= (1_E - 1_F)^+\\
  1_{E\cap F} &= 1_E - 1_{E\setminus F}\\
  1_{E\cup F} &= 1_F + 1_{E\setminus F}
\end{align*}
e usare le proprietà di reticolo.
Osservando poi che 
\[
 1_E + 1_F = 1_{E\cup F} + 1_{E\cap F}
\]
si ottiene la formula
\[
\abs{E} + \abs{F} = \abs{E\cup F} + \abs{E\cap F}.
\]

Supponiamo ora che $E$ sia limitato
e dimostriamo che $\partial E$ ha misura nulla.
Bisogna osservare 
che $\sup 1_{\partial E}(Q^N_{\vec k}) = 1$ 
significa che $\partial E\cap Q^N_{\vec k} \neq \emptyset$.
Ma allora c'è un punto $\vec x\in Q^N_{\vec k}$ tale che ogni intorno 
di $\vec x$ contiene sia punti di $E$ che punti di $\RR^n\setminus E$.
Se il punto $\vec x$ è interno a $Q^N_{\vec k}$ si ha
\[
  \sup 1_{\partial E}(Q^N_{\vec k}) > \inf 1_{\partial E}(Q^N_{\vec k}).
\]
Se ciò non fosse vero significa che 
$Q^N_{\vec k}$ è interamente contenuto in $E$ o in $\RR^n\setminus E$
e che $\vec x$ sta sulla frontiera del cubetto.
Ma allora la disuguaglianza precedente risulta valida 
per un cubetto $Q^N_{\vec k'}$ adiacente a $Q^N_{\vec k}$.
Dunque si deve avere 
\[
  S^N(1_{\partial E})\le 3^n (S^N(1_E)-S_N(1_E))
\]
(dove $3^n-1$ è il numero di cubetti adiacenti ad un cubetto fissato)
e per $N\to +\infty$ si ottiene 
\[
  S^*(1_{\partial E}) \le 3^n (S^*(1_E)-S_*(1_E)) = 0, 
  \qquad 
  S_*(1_{\partial E})\ge 0.
\]
Dunque $\partial E$ è misurabile e $\abs{\partial E} = 0$.

Se $E$ non è limitato si ripete l'osservazione precedente alle intersezioni 
$E\cap [-R,R]^n$ per ottenere lo stesso risultato.

Visto che $\bar A = A \cup \partial A$, se $A$ è misurabile anche $\bar A$ è 
misurabile e si ha $\abs{\bar A} = \abs{A\setminus \partial A} + \abs{\partial A}
= \abs{A \setminus \partial A} \le \abs{A} \le \abs{\bar A}$,
da cui $\abs{\bar A}=\abs{A}$.
\end{proof}

\begin{theorem}[significato geometrico dell'integrale]
Se $f\colon A\subset \RR^n \to \RR$ definiamo la parte positiva dell'epigrafico di $f$
come:
\[
  E_f = \ENCLOSE{(\vec x,y)\in A\times \RR \colon 0 \le y \le f(\vec x)} \subset \RR^{n+1}.
\]
Se $A\subset \RR^n$ è PJ-misurabile ed $f$ è localmente integrabile 
allora $E_{f^+}$ ed $E_{f^-}$ sono localmente misurabili e se non
hanno entrambi misura infinita si ha 
\[
  \int_A f = \abs{E_{f^+}} - \abs{E_{f^-}}.
\]
\end{theorem}

\begin{proof}
Supponiamo $f\ge 0$, $f$ limitata, $A$ limitato PJ-misurabile.
In tal caso si può osservare che 
per ogni cubetto $n$-dimensionale che tocca la base $A$, 
i cubetti $(n+1)$-dimensionali che coprono il grafico di $f$ 
sono al massimo due in più rispetto all'altezza della colonna, 
ovvero del $\sup$ di $f$ 
sul cubetto base. Per i cubetti interni si ha invece l'opposto. Cioè:
\begin{align*}
    0 & \le S^N(1_{E_{f^+}}) - S^N(f) \le 2\cdot 2^{-N} S^N(1_A),\\
    0 & \le S_N(f) - S_N(1_{E_{f^+}}) \le 2\cdot 2^{-N} S^N(1_A)
\end{align*}
e, per $N\to +\infty$, si ottiene 
\begin{align*}
  0 &\le S^*(1_{E_{f^+}}) - S^*(f) \le 0\cdot S^*(1_A), \\
  0 &\le S_*(f) - S_*(1_{E_{f^+}}) \le 0\cdot S^*(1_A), 
\end{align*}
ovvero $\abs{E_{f^+}}=\int f$, come volevamo dimostrare.

Gli altri casi vengono di conseguenza separando parte positiva e parte negativa 
e passando al limite nei troncamenti.
\end{proof}

\begin{theorem}[integrabilità funzioni continue]
Sia $A\subset \RR^n$ un insieme chiuso PJ-misurabile e $f\colon A \to \RR$ una funzione 
continua. Allora $f$ è localmente R-integrabile. 
Se inoltre $\abs{A}<+\infty$ e $f$ 
è limitata, allora $f$ è R-integrabile.
\end{theorem}
%
\begin{proof}
\emph{Passo 1: supponiamo $A$ compatto RJ-misurabile, 
$f$ continua, limitata, non negativa.}
Per il teorema di Weierstrass esiste $M>0$ tale che $0\le f(x) \le M$ per 
ogni $\vec x \in A$.
Fissiamo $\eps>0$.
Per il teorema di Heine-Cantor esiste $\delta>0$ tale 
che se $\vec x_1,\vec x_2\in A$ e $\abs{\vec x_1 -\vec x_2}<\delta$ 
allora $\abs{f(\vec x_1)-f(\vec x_2)}< \eps$. 
Scegliamo $N\in \NN$ abbastanza grande in modo che si abbia $2^{-N}\sqrt n < \delta$
e in modo che $S^N(1_{\partial A})<\eps$.
Allora si ha
\[
  S^N(f) - S_N(f) 
  = 2^{-nN}\sum_{\vec k} (\max f(Q^N_{\vec k})-\min f(Q^N_{\vec k})) 
  \le 2^{-nN}\eps S^N(1_A).
\]
Visto che $f$ è limitata e $A$ è limitato, $S^N(f)$ è limitata in $N$ 
e facendo tendere $N\to+\infty$ si ottiene 
dunque $S^*(f)-S_*(f) \le 0$ da cui $S^*(f)=S_*(f)$ come volevamo dimostrare.

\emph{Passo 2: supponiamo $A$ chiuso PJ-misurabile, $f$ continua, non negativa.} 
Si applica il passo precedente alla troncata $f_R$.

\emph{Passo 3: caso generale.}
Si applica il passo precedente alle parti 
positiva e negativa $f^+$, $f^-$ di $f$.

\emph{Passo 4: supponiamo $\abs{A}<+\infty$ e $f$ limitata.}  
In tal caso esiste $M$ tale che $\abs{f(\vec x)}\le M$ per 
ogni $x\in A$. 
Per la monotonia dell'integrale si ottiene 
\[
  \abs{\int_A f} 
  \le \int_A \abs{f}
  \le \int_A M = M \cdot \abs{A} < +\infty.  
\]
\end{proof}

\begin{theorem}[Fubini-Tonelli]
Sia $f\colon \RR^n\times \RR^m \to \RR$ Riemann integrabile
tale che per ogni $\vec x\in \RR^n$ la funzione $\vec y \mapsto f(\vec x,\vec y)$ 
risulti anch'essa R-integrabile su $\RR^m$.
Allora 
\[
  \int_{\RR^{n+m}} f = \int_{\RR^n} \Enclose{\int_{\RR^m} f(\vec x, \vec y)\, d\vec y}\, d\vec x.
\]
\end{theorem}

\begin{proof}
Supponiamo che $f$ sia una funzione R-integrabile, 
limitata con supporto limitato.
Allora posto 
\[
    g(\vec x) = \int f(\vec x, \vec y)\, d\vec y\\
\]
si ha
\begin{align*}
    S^N(g)
    &= 2^{-nN} \sum_{\vec k\in \ZZ^n} 
        \sup_{\vec x\in Q^N_{\vec k}}
            \int f(\vec x,\vec y)\, dy\\
    &\le 2^{-nN} \sum_{\vec k\in \ZZ^n} 
        \sup_{\vec x\in Q^N_{\vec k}}
                S^N(f(\vec x,\cdot))\\
    &\le 2^{-nN} \sum_{\vec k\in \ZZ^n} 
        \sup_{\vec x\in Q^N_{\vec k}}
            2^{-mN} \sum_{\vec j \in \ZZ^m} 
                \sup_{\vec y \in Q^N_{\vec j}} 
                    f(\vec x, \vec y) \\
    &\le 2^{-(n+m)N} \sum_{\vec k\in \ZZ^n} 
        \sum_{\vec j \in \ZZ^m} 
            \sup_{f(Q^N_{\vec k}\times Q^N_{\vec j})}\\
    &= S^N(f)
\end{align*}
ed analogamente si trova
\[
    S_N(g) \ge S_N(f).
\]
Passando al limite otteniamo quindi 
\[
   S_*(f) \le S_*(g) \le S^*(g) \le S^*(f)
\]
e dunque se $f$ è integrabile (in $n+m$ variabili) anche $g$ è integrabile
(in $n$ variabili) 
e risulta $\int f = \int g$, come volevamo dimostrare.

Gli altri casi si ottengono di conseguenza.
\end{proof}

\begin{theorem}[significato geometrico del determinante]
  \label{th:geometria_determinante}
Sia $L\colon \RR^n\to \RR^n$ una applicazione lineare affine
\[
  L(\vec x) = M\vec x + \vec q.
\]
Allora se $E\subset \RR^n$ è PJ-misurabile si ha 
\[
  \abs{L(E)} = \abs{\det M} \cdot \abs{E}.
\]
\end{theorem}
\begin{proof}
Si vedano appunti di Analisi Uno.
\end{proof}

\begin{theorem}[formula dell'area]
Sia $\phi\colon [0,1]^n \to \RR^n$
una funzione iniettiva, di classe $C^1$, tale che $\det D\phi(x)\neq 0$ per ogni $x\in [0,1]^n$.

Allora $\phi(E)$ è misurabile e si ha
\begin{equation}\label{eq:area}
 \abs{\phi(E)} = \int_E \abs{\det D\phi}.
\end{equation}
\end{theorem}
%
\begin{proof}
Poniamo $Q=[0,1]^n$.
\emph{Passo 1: dimostriamo che $\phi(Q)$ è misurabile.}
Per il teorema di invertibilità locale, ogni punto $\vec x \in (0,1)^n$
ha un intorno che viene mandato, tramite $\phi$, in un intorno del punto di arrivo $\phi(\vec x)$.
Questo significa che punti interni a $Q$ vengono mandati in punti interni a $\phi(Q)$
e di conseguenza 
\[
  \partial \phi(Q) \subset \phi(\partial Q).
\]
Dunque per ogni $N$ sappiamo che $\partial \phi(Q)$ è contenuta nell'immagine 
dei cubetti $Q^n_{\vec k}\subset [0,1]^n$ che toccano la frontiera di $[0,1]^n$.
Il numero di questi cubetti è di poco inferiore a $2n\cdot 2^{(n-1)N}$.
Visto che $D\phi$ è continua, per Weierstrass esiste $L>0$ tale che $\Abs{D\phi(\vec x)} \le L$ per ogni $\vec x\in Q$.
Dalla disuguaglianza di Lagrange troviamo la condizione di Lipschitz:
\[
\abs{\phi(\vec x) - \phi(\vec x')} \le L \abs{\vec x - \vec x'}, \qquad \forall \vec x,\vec x' \in Q.
\]
Fissato un cubetto $Q^N_{\vec k}\subset Q$, 
deduciamo che il diametro di $\phi(Q^N_{\vec k})$ si stima con $L$ volte il diametro di $Q^N_{\vec k}$,
ovvero con $L \sqrt n 2^{-N}$.
Dunque $\phi(Q^N_{\vec k})$ può essere coperto con $(L\sqrt n +1)^n$ cubetti di lato $2^{-N}$.
Da questo deduciamo che 
\[
  S^N(1_{\partial \phi([0,1]^n)}) \le 2n\cdot 2^{(n-1)N}\cdot (L\sqrt n + 1)^n \cdot S^N(1_{\partial [0,1]^n})
\]
e quindi, facendo tendere $N\to +\infty$, troviamo che 
$S^*(1_{\partial \phi(Q)}) = 0 \cdot S^*(1_{\partial Q})=0$.
Visto che ovviamente $S_*(1_{\partial \phi(Q)}) \ge 0$, 
si ottiene che $\partial \phi(Q)$ è misurabile 
di misura nulla. Dunque $\phi(Q)$ è misurabile.

\emph{Passo 2 (lemma del cubo interno/esterno): 
se esiste $\eps\in(0,\frac 1 4)$ tale che 
\[
  \abs{\phi(\vec x)-\vec x} \le \eps, \qquad \forall \vec x \in Q
\]
allora }
\[
  (1-2\eps)^n
  \le \abs{\phi(Q)}
  \le (1+2\eps)^n.
\]
Chiaramente 
\[
    \phi(Q) \subset [-\eps,1+\eps]^n
\]
perché ogni punto di $Q$ viene spostato al massimo di $\eps$. 
Quindi si ottiene, per monotonia della misura,
\[
    \abs{\phi(Q)} \le (1+2\eps)^n.
\]
Per mostrare l'altra disuguaglianza  
\[
   \abs{\phi(Q)} \ge (1-2\eps)^n
\]
basterà dimostrare che vale anche 
\[
  \phi(Q) \supset (\eps,1-\eps)^n.
\]
Questo risulta meno ovvio.
Supponiamo per assurdo che esista $y \in (\eps,1-\eps)^n \setminus \phi(Q)$.
Consideriamo il punto $\vec p=(\frac 12, \dots, \frac 12)$ centro 
del cubo unitario $Q$ e sia $\vec q=\phi(\vec p)$ la sua immagine. 
Visto che $\abs{\vec q-\vec p} = \abs{\phi(\vec p)-\vec p} \le \eps$ 
e $\eps \le \frac 1 4$ possiamo garantire che
$\vec q \in (\eps,1-\eps)^n$.
Dunque l'intero segmento $\sigma = [\vec q,\vec y]$ è contenuto 
in $(\eps,1-\eps)^n$ essendo quest'ultimo un convesso.
Visto che $\vec q\in \phi(Q)$ mentre $\vec y \notin \phi(Q)$,
deve esistere un punto $\vec z\in \sigma\cap \partial \phi(Q)$
(esercizio~\ref{ex:punto-di-frontiera}).
Ma abbiamo già mostrato (Passo 1) che $\partial (\phi(Q)) \subset \phi(\partial Q)$
dunque $\vec z=\phi(\vec x)$ con $\vec x \in \partial Q$. 
Ma allora $\abs{\vec z-\vec x} = \abs{\phi(\vec x)- \vec x} \le \eps$ 
e dunque $\vec z\not \in (\eps,1-\eps)^n$, 
contraddicendo $\vec z \in \sigma \subset (\eps,1-\eps)^n$.

\emph{Passo 3: se esiste $\eps>0$ tale che 
\[
  \Abs{ D\phi(\vec x) - I} \le \frac{\eps}{\sqrt n} \qquad \forall \vec x \in [0,1]^n,
\]
(dove $I$ è la matrice identità)
allora
\[
 (1-2\eps)^n \le \abs{\phi([0,1]^n)} \le (1+2\eps)^n.
\]
}
Se consideriamo la funzione $\psi(x) = \phi(x) - x$ osserviamo che $\psi(0)=\phi(0)$ e
\[
\Abs{D\psi(\vec x)} = \Abs{D\phi(\vec x) -I} \le \frac{\eps}{\sqrt n}.
\]
Applicando la disuguaglianza di Lagrange si ottiene dunque:
\[
\abs{\phi(x) - \phi(0) - x} = \abs{\psi(x)-\psi(0)} \le \frac \eps {\sqrt n} \abs{x} \le \eps
\]
e la conclusione segue dal passo precedente applicato alla funzione $\phi(\vec x) - \phi(0)$.

\emph{Passo 4: supponiamo che esista una matrice invertibile $M$ e un $\eps>0$ tali che
\[
  \Abs{D\phi(\vec x) - M} \le \frac{\eps}{\Abs{M^{-1}} \sqrt n}, \qquad \forall \vec x \in [0,1]^n.
\]
allora
\[
(1-2\eps)^n \le \frac{\abs{\phi([0,1]^n)}}{\det M} \le (1+2\eps)^n.
\]}

Consideriamo la funzione $\psi(\vec x) = M^{-1} \phi(\vec x)$ per la quale si ha  
\[
  \Abs{D\psi(\vec x)-I} 
  = \Abs{M^{-1} (D\phi(\vec x)-M)} 
  \le \Abs{M^{-1}} \cdot \Abs{D\phi(\vec x)-M}
  \le \frac{\eps}{\sqrt n}.
\]
Applicando il passo precedente alla funzione $\psi$ otteniamo 
\[
(1-2\eps)^n \le \abs{\psi([0,1]^n)} \le (1+2\eps)^n.
\]
Ricordando il Teorema \ref{th:geometria_determinante} si ha
\[
  \abs{\psi([0,1]^n)} = \abs{M^{-1}(\phi([0,1]^n))}
  = \abs{\det M^{-1}} \cdot \abs{\phi([0,1]^n)}
  = \frac{\abs{\phi([0,1]^n)}}{\abs{\det M}}
\]
che conclude il passo corrente.
\end{proof}


\begin{theorem}[cambio di variabile]
Sia $E\subset \RR^n$ un chiuso misurabile, $\phi\colon E \to \RR^n$
una funzione di classe $C^1$, 
$\phi$ iniettiva se ristretta alla parte interna di $E$,
$f\colon \phi(E)\to \RR$ una funzione 
continua. Allora
\[
    \int_E f(\phi(\vec x))\abs{\det D \phi(\vec x)}\, d \vec x
    =
    \int_{\RR^n} f(\vec y) \, d \vec y.
\]
\end{theorem}

\begin{proof}
\emph{Caso 1: supponiamo $E$ compatto.}
Allora $\Abs{D\phi}$ ha massimo $L$ su $E$, e $f$ ha massimo $M$ 
su $\phi(E)$. 

Visto che $D\phi$ è continua su $E$, è anche uniformemente continua e 
usando il teorema di Lagrange si può mostrare che per ogni $\eps>0$ 
esiste $\delta>$ tale che
\[
 \abs{\vec x - \vec y}<\delta 
 \implies 
 \abs{\phi(\vec x)-\phi(\vec y)-D\phi(\vec y)(\vec x - \vec y)}
 \le \eps (\vec x - \vec y).
\]
Sia dato $\eps\in(0,1)$.
Sia $N\in \NN$ abbastanza grande in modo che, posto
\begin{align*}
  K&=\ENCLOSE{\vec k\in \ZZ^n\colon Q^N_{\vec k}\subset E},\\
  K'&=\ENCLOSE{\vec k\in \ZZ^n\setminus K \colon Q^N_{\vec k}\cap E\neq \emptyset}.
\end{align*}
e posto, per $\vec k \in K$,dove 
\[
  \deg(\phi,\vec y) = \sum_{\phi(\vec x)=\vec y} \sgn \det \phi(\vec x).
\]

\begin{gather*}
  \vec x^N_{\vec k} = 2^{-N}\vec k,\\
  f^N_{\vec k} = f(\phi(\vec x^N_{\vec k})),\\ 
  \Phi^N_{\vec k} = D\phi(\vec x^N_{\vec k}), 
\end{gather*}
si abbia
\begin{gather*}
  M L^d (S^N(1_E)-S_N(1_E)) < \eps,\\
  \forall \vec k\in K \colon \forall \vec x \in Q^N_{\vec k}\colon
    \abs{f(\phi(\vec x))-f^N_{\vec k}} < \eps,\\
  \forall \vec k\in K \colon \forall \vec x \in Q^N_{\vec k}\colon
    \abs{\phi(\vec x) - \phi(\vec x^N_{\vec k})-\Phi^N_{\vec k}[\vec x-\vec x^N_{\vec k}]} < \eps 2^{-N}.
\end{gather*}

Per il Lemma, sappiamo che se $\vec k \in K$ allora $\phi(Q^N_{\vec k})$ è PJ-misurabile 
e si ha 
\[
 \abs{\phi(Q^N_{\vec k})} 
 \le (L\sqrt n)^d 2^{-nN}
\]

\begin{align*}
  \abs{\int_{\phi(E)} f(\vec y)\, d\vec y 
  - \sum_{\vec k \in K} \int_{\phi(Q^N_{\vec k})} f(\vec y)\, d\vec y}
  &\le M \cdot \#K' \cdot (\sqrt n K 2^{-N})^n \\
  &\le M (S^N(1_E)-S_N(1_E)) (\sqrt n K)^n
  \le \eps 
\end{align*}

\begin{align*}
\sum_{\vec k \in K}\int_{\phi(Q^N_{\vec k})} \abs{f(\vec y) - f^N_{\vec k}}\, d\vec y
&\le  \abs{\phi(E)} \eps 
\end{align*}

Ora notiamo che siccome per $\vec x \in Q^N_{\vec k}$ si ha 
\[
  \abs{
    \phi(\vec x) - \phi(\vec x^N_{\vec k}) 
    - \Phi^N_{\vec k}(\vec x - \vec x^N_{\vec k})} \le \eps 2^{-N}
\]
risulta 
\begin{align*}
 \phi(Q^N_{\vec k}) - \phi(\vec x^N_{\vec k}) \subset \Phi(Q^N_{\vec k})+B_{\eps 2^{-N}}
\end{align*}

\begin{align*}
\sum_{\vec k \in K}
  \int_{\phi(Q^N_{\vec k})} f^N_{\vec k}\, d\vec y 
= \sum_{\vec k in K} f^N_{\vec k} \abs{\phi(Q^N_{\vec k})}
\end{align*}

***

Per Taylor per ogni $N\in \NN$ e $\vec k\in \ZZ^n$,
posto ${\vec x}^N_{\vec k} = 2^{-N}\vec k$ e $\Phi^N_{\vec k}=D\phi({\vec x}^N_{\vec k})$,
si ha 
\begin{align*}
  \phi(\vec x) &= \phi({\vec x}^N_{\vec k}) + \Phi^N_{\vec k} (\vec x - {\vec x}^N_{\vec k}) + \eps 2^{-Nn}\\
  f(\phi(\vec x)) &= F^N_{\vec k} + \eps\\
  \abs{f(\vec x)} & \le M\\
  \abs{\det D\phi(\vec x)} & \le M\\
\end{align*}
cosicché
\begin{align}
  \abs{
    F^N_{\vec k}  2^{-Nn} \det \Phi^N_{\vec k}
    - \int_{Q^N_{\vec k}} f(\phi(\vec x)) \det D\phi(\vec x)\, dx
  }
  &\le \int_{Q^N_{\vec k}}  \Enclose{M \eps 2^{-nN} + \eps M}\, dx \\
  &\le 2M\eps 2^{-nN}. 
\end{align}
ma 
\begin{align*}
    F^N_{\vec k} 2^{-nN} \det \Phi^N_{\vec k} 
    &= F^N_{\vec k}\abs{\phi(Q^N_{\vec k})}.
\end{align*}
e
\begin{align*}
  \int_{\phi(Q^N_{\vec k})} f(\vec y) \deg(\phi,\vec y)\, dy 
  &=
  \sum
\end{align*}
\end{proof}