\chapter{integrali multipli}

\section{integrale di Riemann}
Se $x,y\in \RR$ definiamo:
\[
    x \vee y \defeq \max\ENCLOSE{x,y},
    \quad
    x \wedge y \defeq \min\ENCLOSE{x,y},
    \quad
    x^+ \defeq x \vee 0,
    \quad
    x^- \defeq (-x)^+ \defeq -(x\wedge 0).
\]
Si ha
\[
    x^+,x^-\ge 0,
    \quad
    x = x^+ - x^-,
    \quad 
    \abs{x} = x^+ + x^-,
    \quad
    x^+ = \frac{\abs{x}+x}{2},
    \quad
    x^- = \frac{\abs{x}-x}{2}.
\]

Sia $A\subset \RR^n$ e $f\colon A \to \RR$. 
Definiamo il \emph{supporto} di $f$ come 
\[
 \spt f 
    \defeq \overline{\ENCLOSE{\vec x \in A \colon f(\vec x) \neq 0}}
    = \overline{f^{-1}(\RR\setminus\ENCLOSE{0})}.
\]

Dato $A\subset \RR^n$ definiamo la \emph{funzione caratteristica}
di $A$ come $1_A\colon \RR^n \to \RR$,
\[
  1_A(\vec x) \defeq \begin{cases}
    1 & \text{se $\vec x \in A$,}\\
    0 & \text{altrimenti.}
  \end{cases}
\]

Per $\vec k \in \ZZ^d$ e $N\in \NN$ definiamo 
il cubetto 
\[
  Q^N_{\vec k} \defeq \frac{\vec k + [0,1]^d}{2^N}.
\]
Si tratta di un cubetto di lato $2^{-N}$ con gli spigoli paralleli 
agli assi cartesiani e con un vertice nel punto $2^{-N} \vec k$.
Il volume di $Q^N_{\vec k}$ sarà quindi $2^{-nN}$.
Useremo questi cubetti per misurare gli insiemi.

\begin{definition}[integrale di Riemann]
  \label{def:integrale_riemann}
Sia $f\colon \RR^n \to \RR$ una funzione limitata, 
con supporto limitato.
Definiamo le somme superiori e inferiori sulla $N$-esima suddivisione in cubi:
\begin{align*}
    S^N(f) &\defeq 2^{-nN}\sum_{\vec k\in \ZZ^n} \sup f(Q^N_{\vec k}), \\
    S_N(f) &\defeq 2^{-nN}\sum_{\vec k\in \ZZ^n} \inf f(Q^N_{\vec k}).
\end{align*}
Visto che il supporto di $f$ è limitato, queste somme sono in realtà 
somme finite. Visto che $f$ è limitata, ogni addendo è finito e dunque 
$S^N(f), S_N(f) \in \RR$ per ogni $N\in \NN$.
Osservando che $N\mapsto S^N(f)$ è decrescente 
mentre $N\mapsto S_N(f)$ è crescente, definiamo
l'integrale superiore e inferiore:
\begin{align*}
    S^*(f) &\defeq \inf_N S^N(f) = \lim_{N\to +\infty} S^N(f) \in \RR, \\
    S_*(f) &\defeq \sup_N S_N(f) = \lim_{N\to +\infty} S_N(f) \in \RR.
\end{align*}
Se $S^*(f)=S_*(f)$ diremo che $f$ è Riemann integrabile (o R-integrabile)
e scriveremo
\[
  \int\!\! f = S^*(f) = S_*(f).
\]

Volendo rendere esplicita la variabile di integrazione, potremo
usare la notazione equivalente
\[
  \int\!\! f(x)\, dx \defeq \int\!\! f.
\]
\end{definition}

\begin{theorem}[monotonia dell'integrale]
Siano $f,g\colon \RR^n\to\RR$ due funzioni limitate con supporto limitato.
Se $f$ e $g$ sono $R$-integrabili si ha:
\[
  f \le g \implies \int\!\! f \le \int\!\! g.
\]
\end{theorem}
%
\begin{proof}
Se $f\le g$, per ogni $N\in \NN$ e ogni $\vec k \in \ZZ^n$ si ha 
$\sup f(Q^N_{\vec k}) \le \sup g(Q^N_{\vec k})$
da cui 
$S^N(f) \le S^N(g)$.
Passando al limite per $N\to +\infty$ si ottiene
$S^*(f) \le S^*(g)$
che è la tesi se $f$ e $g$ sono R-integrabili.
\end{proof}

\begin{theorem}[linearità dell'integrale]
Siano $f,g\colon \RR^n \to \RR$ funzioni limitate con supporto limitato.
Se $f,g$ sono R-integrabili allora $f+g$ è R-integrabile e si ha 
\[
    \int (f+g) = \int\!\! f + \int\!\! g.
\]
Se $f$ è R-integrabile e $c\in \RR$ è costante, anche $c f$ è R-integrabile 
e si ha 
\[
    \int\!\! c f = c \!\int\!\! f.
\]
\end{theorem}
\begin{proof}
Per ogni $\vec k\in\ZZ^n$ e ogni $N\in \NN$
si ha
\begin{align*}
  \sup\, (f+g)(Q^N_{\vec k}) &\le \sup f(Q^N_{\vec k}) + \sup g(Q^N_{\vec k})\\
  \inf\, (f+g)(Q^N_{\vec k}) &\ge \inf f(Q^N_{\vec k}) + \inf g(Q^N_{\vec k}).
\end{align*}
Dunque si ottiene
\[
    S^N(f+g) \le S^N(f) + S^N(g),  \qquad
    S_N(f+g) \ge S_N(f) + S_N(g), 
\]
e passando al limite per $N\to +\infty$ 
\[
   S_*(f) + S_*(g) \le S_*(f+g) \le S^*(f+g) \le S^*(f) + S^*(g).
\]
Se $f$ e $g$ sono R-integrabili si ottiene dunque che anche $f+g$ 
è R-integrabile e inoltre $\int (f+g) = \int\! f + \int\! g$.

Se $c \ge 0$ si ha 
\[
   S^N(c f) = c S^N(f), \qquad S_N(c f) = c S_N(f)
\]
e quindi se $f$ è R-integrabile si conclude 
che $c f$ è R-integrabile e $\int\! c f = c \int\! f$.

Se $c = -1$ si osserva che per ogni $N\in \NN$, $\vec k\in \ZZ^n$, 
si ha 
\[
   \sup_{Q^N_{\vec k}} (-f) = -\inf_{Q^N_{\vec k}} f,
   \qquad 
   \inf_{Q^N_{\vec k}} (-f) = -\sup_{Q^N_{\vec k}} f,
\]
da cui 
\[
    S^N(-f) = -S_N(f), \qquad S_N(-f) = -S^N(f)
\]
e quindi, per $N\to +\infty$ si ottiene 
\[
  S^*(-f) = -S_*(f), \qquad S_*(-f) = - S^*(f).
\]
Se $f$ è R-integrabile deduciamo che anche $-f$ lo è e si ha 
$\int (-f) = -\int\! f$.

Combinando i casi precedenti si ottiene il caso generale $c \in \RR$.
\end{proof}

\begin{theorem}[proprietà di reticolo]
Sia $f\colon \RR^n \to \RR$ una funzione limitata con supporto limitato.
Se $f$ è R-integrabile allora anche $\abs{f}$, $f^+$ e $f^-$ sono 
R-integrabili e si ha 
\[
  \abs{\int\!\! f} \le \int\! \abs f = \int\!\! f^+ + \int\!\! f^-, 
  \qquad 
 \int\!\! f = \int\!\! f^+ - \int\!\! f^-.
\]
\end{theorem}
\begin{proof}
Dalla proprietà generale 
\[
  \abs{\rule{0pt}{1.1em}\abs{f(\vec x)} - \abs{f(\vec y)}} \le \abs{f(\vec x) - f(\vec y)}
\]
si ottiene che per ogni $N\in \NN$, $\vec k \in \ZZ^n$, vale
\[
  \sup_{Q^N_{\vec k}} \abs{f} - \inf_{Q^N_{\vec k}} \abs{f}
  \le 
  \sup_{Q^N_{\vec k}} f - \inf_{Q^N_{\vec k}} f
\]
e quindi 
\[
   S^N(\abs f) - S_N(\abs f) \le S^N(f) - S_N(f)
\] 
da cui, passando al limite per $N\to +\infty$,
\[
  S^*(\abs f) - S_*(\abs f) \le S^*(f) - S_*(f)
\]
per cui se $f$ è R-integrabile anche $\abs{f}$ lo è.

Osserviamo ora che $f^+=\frac{\abs f + f}{2}$ e $f^-=\frac{\abs f - f}{2}$ 
e dunque per linearità se $f$ è R-integrabile avendo già dimostrato 
che anche $\abs f$ è R-integrabile, si ottiene che pure $f^+
$ e $f^-$ sono R-integrabili ed essendo $f=f^+-f^-$, $\abs f = f^+ + f^-$ si ottiene 
\[
  \int\!\! f = \int\!\! f^+ - \int\!\! f^-, \qquad 
  \int\! \abs f = \int\!\! f^+ + \int\!\! f^-.
\]
Per la convessità del valore assoluto si ottiene infine
\[
 \abs{\int\!\! f} = \abs{\int\!\! f^+ - \int\!\! f^-} 
 \le \abs {\int\!\! f^+} + \abs{\int\!\! f^-}
 = \int\!\! f^+ + \int\!\! f^- = \int\! \abs f.
\]
\end{proof}

\begin{definition}[integrale di Riemann generalizzato]
Se $f\colon \RR^n \to [0,+\infty]$ è una funzione (non negativa), per ogni $R>0$ 
possiamo definire il troncamento $f_R$ 
\[
  f_R = (f \cdot 1_{[-R,R]^n}) \vee R.
\]
La funzione $f_R$ è limitata ed ha supporto limitato.
Se $f_R$ è Riemann-integrabile per ogni $R>0$ diremo che $f$ è localmente R-integrabile 
e visto che $R\mapsto f_R(\vec x)$ è crescente possiamo definire
\[
 \int\!\! f \defeq \lim_{R\to+\infty} \int\!\! f_R \in [0,+\infty].
\]
Se $\int\!\! f<+\infty$ diremo che $f$ è R-integrabile (avendo supposto $f\ge 0$)

Se $f\colon \RR^n\to \RR$ possiamo scrivere $f= f^+-f^-$ con $f^+,f^-$ funzioni non negative.
Se $f^+$ e $f^-$ sono localmente R-integrabili e se 
$\int\!\! f^+$ e $\int\!\! f^-$ non sono entrambi infiniti possiamo definire 
\[
  \int\!\! f = \int\!\! f^+ - \int\!\! f^- \in [-\infty,+\infty].
\]
Se $\int\!\! f$ è finito diremo che $f$ è $R$-integrabile.
\end{definition}

\begin{definition}
Dato $E\subset \RR^n$ potremo definire la misura (di Peano-Jordan) esterna 
e interna di $E$ come 
\begin{align*}
 m^N(E) &\defeq S^N(1_E), & m^*(E) &\defeq S^*(1_E),\\
 m_N(E) &\defeq S_N(1_E), & m_*(E) &\defeq S_*(1_E).
\end{align*}

Diremo che $E$ è \emph{Peano-Jordan misurabile} 
(PJ-misurabile) se la funzione caratteristica $1_E$ è localmente 
R-integrabile,
ovvero se $m^*(E\cap[-R,R]^n) = m_*(E\cap[-R,R]^n)$ per ogni $R>0$.
In tal caso definiamo la misura di $E$ come 
\[
\abs{E} \defeq \int\!\! 1_E \in [0,+\infty] = \lim_{R\to+\infty} \abs{E\cap[-R,R]^n}.
\]


Se $A\subset \RR^n$ è PJ-misurabile e $f\colon A \to \RR$
è una funzione qualunque, possiamo considerare l'estensione 
$f_A\colon \RR^n \to \RR$ definita da 
\[
 f_A(\vec x) = \begin{cases}
    f(\vec x) & \text{se $x\in A$,}\\
    0 & \text{altrimenti.}
 \end{cases}
\]
Diremo che $f$ è localmente R-integrabile in $A$ se $f_A$ 
è localmente R-integrabile. 
Scriveremo:
\[
 \int_A f = \int\!\! f_A
\]
se $\int f_A$ è definito.
Se inoltre $\int_A f$ è finito diremo che $f$ è R-integrabile
su $A$.

Volendo rendere esplicita la variabile di integrazione, potremo 
scrivere 
\[
  \int_A f(x)\, dx \defeq \int_A f.
\]
\end{definition}

\begin{remark}[caso $n=1$]
Nel caso $n=1$ se $f\colon [a,b]\to \RR$ è una funzione limitata,
si osserva che la suddivisione in cubetti di lato $2^{-N}$ induce una partizione 
su $[a,b]$ e la definizione di Riemann-integrabilità data sopra coincide con la 
usuale Riemann-integrabilità definita tramite suddivisioni arbitrarie.

Nel caso $f\colon(a,b)\to \RR$ con $(a,b)$ non limitato oppure $f$ non limitata,
la definizione data sopra corrisponde invece alla \emph{assoluta convergenza} 
dell'integrale improprio.
\end{remark}

\begin{theorem}[algebra dei misurabili]
Se $E,F\subset \RR^n$ sono PJ-misurabili allora anche $\bar E$, $\partial E$, 
$E\setminus F$, $E\cap F$ ed $E\cup F$ sono PJ-misurabili.
Inoltre si ha 
\[
    \abs{E\cup F} + \abs{E\cap F} = \abs{E} + \abs{F},
    \qquad
    \abs{\partial E} = 0.
\]
\end{theorem}
\begin{proof}
Per ottenere la misurabilità di $E\setminus F$ e di $E\cap F$, basta osservare che 
\begin{align*}
  1_{E\setminus F} &= (1_E - 1_F)^+\\
  1_{E\cap F} &= 1_E - 1_{E\setminus F}\\
  1_{E\cup F} &= 1_F + 1_{E\setminus F}
\end{align*}
e usare le proprietà di reticolo.
Osservando poi che 
\[
  1_{E\cup F} + 1_{E\cap F} = 1_E + 1_F 
\]
si ottiene la formula
\[
  \abs{E\cup F} + \abs{E\cap F} = \abs{E} + \abs{F}.
\]

Supponiamo ora che $E$ sia limitato
e dimostriamo che $\partial E$ ha misura nulla.
Bisogna osservare 
che $\sup 1_{\partial E}(Q^N_{\vec k}) = 1$ 
significa che $\partial E\cap Q^N_{\vec k} \neq \emptyset$.
Ma allora c'è un punto $\vec x\in Q^N_{\vec k}$ tale che ogni intorno 
di $\vec x$ contiene sia punti di $E$ che punti di $\RR^n\setminus E$.
Se il punto $\vec x$ è interno a $Q^N_{\vec k}$ si ha
\[
  \sup 1_{\partial E}(Q^N_{\vec k}) > \inf 1_{\partial E}(Q^N_{\vec k}).
\]
Se ciò non fosse vero significa che 
$Q^N_{\vec k}$ è interamente contenuto in $E$ o in $\RR^n\setminus E$
e che $\vec x$ sta sulla frontiera del cubetto.
Ma allora la disuguaglianza precedente risulta valida 
per un cubetto $Q^N_{\vec k'}$ adiacente a $Q^N_{\vec k}$.
Dunque si deve avere 
\[
  m^N(\partial E)\le 3^n (m^N(E)-m_N(E))
\]
(in quanto $3^n-1$ è il numero di cubetti adiacenti ad un cubetto fissato)
e per $N\to +\infty$ si ottiene 
\[
  m^*(\partial E) \le 3^n (m^*(E)-m_*(E)) = 0, 
  \qquad 
  m_*(\partial E)\ge 0.
\]
Dunque $\partial E$ è misurabile e $\abs{\partial E} = 0$.

Se $E$ non è limitato si ripete l'osservazione precedente alle intersezioni 
$E\cap [-R,R]^n$ per ottenere lo stesso risultato.

Visto che $\bar A = A \cup \partial A$, se $A$ è misurabile anche $\bar A$ è 
misurabile e si ha $\abs{\bar A} = \abs{A\setminus \partial A} + \abs{\partial A}
= \abs{A \setminus \partial A} \le \abs{A} \le \abs{\bar A}$,
da cui $\abs{\bar A}=\abs{A}$.
\end{proof}

\begin{theorem}[significato geometrico dell'integrale]
Se $f\colon A\subset \RR^n \to \RR$ definiamo la parte positiva dell'epigrafico di $f$
come:
\[
  E_f = \ENCLOSE{(\vec x,y)\in A\times \RR \colon 0 \le y \le f(\vec x)} \subset \RR^{n+1}.
\]
Se $A\subset \RR^n$ è PJ-misurabile ed $f$ è localmente integrabile 
allora $E_{f^+}$ ed $E_{f^-}$ sono localmente misurabili e, se non
hanno entrambi misura infinita, si ha 
\[
  \int_A f = \abs{E_{f^+}} - \abs{E_{f^-}}.
\]
\end{theorem}

\begin{proof}
Supponiamo $f\ge 0$, $f$ limitata, $A$ limitato e PJ-misurabile.
In tal caso si può osservare che 
per ogni cubetto $n$-dimensionale che tocca la base $A$, 
i cubetti $(n+1)$-dimensionali che coprono il grafico di $f$ 
sono al massimo due in più rispetto all'altezza della colonna, 
ovvero del $\sup$ di $f$ sul cubetto base. 
Per i cubetti interni si ha invece l'opposto. Cioè:
\begin{align*}
    0 & \le m^N(E_{f^+}) - S^N(f) \le 2\cdot 2^{-N} m^N(A),\\
    0 & \le S_N(f) - m_N(E_{f^+}) \le 2\cdot 2^{-N} m^N(A)
\end{align*}
e, per $N\to +\infty$, si ottiene 
\begin{align*}
  0 &\le m^*(E_{f^+}) - S^*(f) \le 0\cdot m^*(A), \\
  0 &\le S_*(f) - m_*(E_{f^+}) \le 0\cdot m^*(A), 
\end{align*}
ovvero $\abs{E_{f^+}}=\int\! f$, come volevamo dimostrare.

Gli altri casi vengono di conseguenza separando parte positiva e parte negativa 
e passando al limite nei troncamenti.
\end{proof}

\begin{theorem}[integrabilità funzioni continue]
Sia $A\subset \RR^n$ un insieme chiuso PJ-misurabile e $f\colon A \to \RR$ una funzione 
continua. Allora $f$ è localmente R-integrabile. 
Se inoltre $\abs{A}<+\infty$ e $f$ 
è limitata, allora $f$ è R-integrabile.
\end{theorem}
%
\begin{proof}
\emph{Passo 1: supponiamo $A$ compatto RJ-misurabile, 
$f$ continua, limitata, non negativa.}
Per il teorema di Weierstrass esiste $M>0$ tale che $0\le f(x) \le M$ per 
ogni $\vec x \in A$.
Fissiamo $\eps>0$.
Per il teorema di Heine-Cantor esiste $\delta>0$ tale 
che se $\vec x_1,\vec x_2\in A$ e $\abs{\vec x_1 -\vec x_2}<\delta$ 
allora $\abs{f(\vec x_1)-f(\vec x_2)}< \eps$. 
Scegliamo $N\in \NN$ abbastanza grande in modo che si abbia $2^{-N}\sqrt n < \delta$
e in modo che $m^N(\partial A)<\eps$.
Allora si ha
\[
  S^N(f) - S_N(f) 
  = 2^{-nN}\sum_{\vec k} (\max f(Q^N_{\vec k})-\min f(Q^N_{\vec k})) 
  \le 2^{-nN}\eps m^N(A).
\]
Visto che $f$ è limitata e $A$ è limitato, $S^N(f)$ è limitata in $N$ 
e facendo tendere $N\to+\infty$ si ottiene 
dunque $S^*(f)-S_*(f) \le 0$ da cui $S^*(f)=S_*(f)$ come volevamo dimostrare.

\emph{Passo 2: supponiamo $A$ chiuso PJ-misurabile, $f$ continua, non negativa.} 
Si applica il passo precedente alla troncata $f_R$.

\emph{Passo 3: caso generale.}
Si applica il passo precedente alle parti 
positiva e negativa $f^+$, $f^-$ di $f$.

\emph{Passo 4: supponiamo $\abs{A}<+\infty$ e $f$ limitata.}  
In tal caso esiste $M$ tale che $\abs{f(\vec x)}\le M$ per 
ogni $x\in A$. 
Per la monotonia dell'integrale si ottiene 
\[
  \abs{\int_A f} 
  \le \int_A \abs{f}
  \le \int_A M = M \cdot \abs{A} < +\infty.  
\]
\end{proof}

\begin{theorem}[Fubini-Tonelli]
Sia $f\colon \RR^n\times \RR^m \to \RR$ Riemann integrabile
tale che per ogni $\vec x\in \RR^n$ la funzione $\vec y \mapsto f(\vec x,\vec y)$ 
risulti anch'essa R-integrabile su $\RR^m$.
Allora 
\[
  \int_{\RR^{n+m}} f = \int_{\RR^n} \Enclose{\int_{\RR^m} f(\vec x, \vec y)\, d\vec y}\, d\vec x.
\]
\end{theorem}

\begin{proof}
Supponiamo che $f$ sia una funzione R-integrabile, 
limitata con supporto limitato.
Allora posto 
\[
    g(\vec x) = \int\!\! f(\vec x, \vec y)\, d\vec y\\
\]
si ha
\begin{align*}
    S^N(g)
    &= 2^{-nN} \sum_{\vec k\in \ZZ^n} 
        \sup_{\vec x\in Q^N_{\vec k}}
            \int\!\! f(\vec x,\vec y)\, dy\\
    &\le 2^{-nN} \sum_{\vec k\in \ZZ^n} 
        \sup_{\vec x\in Q^N_{\vec k}}
                S^N(f(\vec x,\cdot))\\
    &\le 2^{-nN} \sum_{\vec k\in \ZZ^n} 
        \sup_{\vec x\in Q^N_{\vec k}}
            2^{-mN} \sum_{\vec j \in \ZZ^m} 
                \sup_{\vec y \in Q^N_{\vec j}} 
                    f(\vec x, \vec y) \\
    &\le 2^{-(n+m)N} \sum_{\vec k\in \ZZ^n} 
        \sum_{\vec j \in \ZZ^m} 
            \sup_{f(Q^N_{\vec k}\times Q^N_{\vec j})}\\
    &= S^N(f)
\end{align*}
ed analogamente si trova $S_N(g) \ge S_N(f)$.

Passando al limite otteniamo quindi 
\[
   S_*(f) \le S_*(g) \le S^*(g) \le S^*(f)
\]
e dunque se $f$ è integrabile (in $n+m$ variabili) anche $g$ è integrabile
(in $n$ variabili) 
e risulta $\int f = \int g$, come volevamo dimostrare.

Gli altri casi si ottengono di conseguenza.
\end{proof}

\begin{theorem}[significato geometrico del determinante]
  \label{th:geometria_determinante}
Sia $L\colon \RR^n\to \RR^n$ una applicazione lineare affine
\[
  L(\vec x) = M\vec x + \vec q.
\]
Allora se $E\subset \RR^n$ è PJ-misurabile si ha 
\[
  \abs{L(E)} = \abs{\det M} \cdot \abs{E}.
\]
\end{theorem}
\begin{proof}
Si vedano appunti di Analisi Uno.
\end{proof}

\begin{lemma}[misura delle immagini Lipschitziane]
  \label{lm:misura-immagine-lipschitz}
Sia $E \subset \RR^n$ e sia $\phi\colon E \to \RR^m$ una 
funzione $L$-lipschitziana, ovvero tale che
\[
  \abs{\phi(\vec x) - \phi(\vec y)} \le L \abs{\vec x - \vec y}, 
  \qquad \forall \vec x,\vec y \in E.
\]
Allora 
\begin{equation}\label{eq:misura-immagine-lipschitz}
   m^*(\phi(E)) \le (L\sqrt n + 1)^m \cdot m^*(E).
\end{equation}

Se $\Omega\subset \RR^n$ è un aperto limitato PJ-misurabile,
se $\phi \colon \bar \Omega\subset \RR^n \to \RR^n$ 
è una funzione $L$-lipschitziana, 
che ristretta ad $\Omega$ è di classe $C^1$ e 
$\det D\phi(\vec x)\neq 0$ per ogni $\vec x\in \Omega$,  
allora 
dunque $\phi(\Omega)$ è PJ-misurabile.
Inoltre si $\partial \phi(\Omega) \subset \phi(\partial \Omega)$
sono misurabili di misura nulla
e $\overline{\phi(\Omega)} = \phi(\bar \Omega)$ è misurabile.
\end{lemma}
%
\begin{proof}
\emph{Prima parte.}
Definiamo il \emph{diametro} di un insieme $A\subset \RR^n$ come
\[
  \diam(A) \defeq \sup_{\vec x,\vec y\in A} \abs{\vec x - \vec y}.
\]
Chiaramente si ha $\diam(\phi(A)) \le L \cdot \diam(A)$
e dunque se $Q^N_{\vec k}$ è un cubetto di lato $2^{-N}$
il suo diametro è $2^{-N}\sqrt n$ e quindi 
il diametro di $F^N_{\vec k} \defeq \phi(Q^N_{\vec k}\cap \Omega)$ 
è al massimo $L \cdot 2^{-N}\sqrt n$. 
Dunque $F^N_{\vec k}$ 
è contenuto in un cubo allineato agli assi cartesiani, di lato 
$L\sqrt n \cdot 2^{-N}$ e tale cubo può essere ricoperto 
con non più di $(L\sqrt n + 1)^m$ cubetti 
di lato $2^{-N}$, da cui
\[
 m^N(\phi(E)) \le 2^m (L\sqrt n + 1)^m \cdot m^N(E).
\]
Passando al limite per $N\to +\infty$ si ottiene~\eqref{eq:misura-immagine-lipschitz}.

\emph{Seconda parte.}
Visto che $\bar \Omega$ è compatto e $\phi$ è continua, 
sappiamo che $\phi(\bar \Omega)$ è compatto e dunque 
$\partial \phi(\Omega)\subset \phi(\bar \Omega)$.
Ma $\phi\in C^1(\Omega)$ e $\det D\phi(\vec x)\neq 0$,
dunque per il teorema di invertibilità locale
possiamo affermare che ogni punto $\vec x\in \Omega$ ha un intorno
che viene mandato, tramite $\phi$, in un intorno del punto di arrivo 
$\phi(\vec x)$.
Questo significa che punti interni a $\Omega$ vengono mandati 
in punti interni a $\phi(\Omega)$
e dunque che la frontiera di $\phi(\Omega)$ è contenuta nell'immagine 
della frontiera di $\Omega$.
Per continuità sappiamo che $\phi(\bar \Omega) \subset \overline{\phi(\Omega)}$ 
e per compattezza sappiamo che $\overline {\phi(\bar \Omega)} = \phi(\bar \Omega)$.
Dunque $\phi(\bar \Omega) = \overline{\phi(\Omega)}$. 

Per concludere che $\phi(\Omega)$ e $\overline{\phi(\Omega)}$ sono misurabili basta dimostrare che 
$\abs{\partial (\phi(\Omega))}=0$ cioè che $m^*(\partial (\phi(\Omega)))= 0$.
Grazie all'osservazione precedente, ci basta dunque dimostrare
che $m^*(\phi(\partial \Omega)) = 0$.
E questo è vero in quanto \eqref{eq:misura-immagine-lipschitz}
ci dice che
\[
 0 \le m^*(\phi(\partial \Omega)) 
 \le (L\sqrt n+1)^n \cdot m^*(\partial \Omega) = 0
\]
se $\Omega$ è PJ-misurabile e $\phi$ è $L$-lipschitziana.
\end{proof}


\begin{theorem}[formula dell'area]
  \label{th:formula-area}
Sia $\Omega\subset \RR^n$ un aperto convesso limitato PJ-misurabile.
Sia $\phi\colon \bar \Omega \to \RR^n$ una funzione di 
classe $C^1$ (significa che $\phi$ può essere estesa ad una funzione $C^1$ su 
un aperto contenente $\bar \Omega$) 
tale che $\phi$ è iniettiva su $\Omega$ 
e $\det D \phi(\vec x)\neq 0$ per ogni $\vec x\in \Omega$.

Allora $\phi(\Omega)$ è PJ-misurabile e si ha 
\[
  \abs{\phi(\Omega)} 
    = \int_{\Omega} \abs{\det D\phi}.
\]

%Sia $Q=[0,1]^n$ e sia $\phi\colon Q \to \RR^n$
%una funzione iniettiva su $(0,1)^n$, di classe $C^1$, tale che $\det D\phi(x)\neq 0$ per ogni $x\in [0,1]^n$.

%Allora $\phi(Q)$ è misurabile e si ha
%\begin{equation}\label{eq:area}
% \abs{\phi(Q)} = \int_{Q} \abs{\det D\phi}.
%\end{equation}
%
%Notiamo che il teorema~\ref{th:cambio-variabili} ci dirà che questo risultato 
%è vero in ipotesi leggermente più generali.
\end{theorem}
%
\begin{proof}
\emph{Passo 1: dimostriamo che $\phi(\Omega)$ è misurabile.}
Essendo $\Omega$ convesso possiamo applicare il teorema~\ref{th:lagrange} di Lagrange 
per ottenere la condizione di Lipschitz:
\[
\abs{\phi(\vec x) - \phi(\vec x')} \le L \abs{\vec x - \vec x'}, \qquad \forall \vec x,\vec x' \in \Omega.
\]
Per continuità la condizione si estende a tutte le coppie $\vec x, \vec x' \in \bar \Omega$.
Possiamo quindi applicare il Lemma~\ref{lm:misura-immagine-lipschitz} 
per ottenre la misurabilità di $\phi(\Omega)$ e di $\phi(\bar \Omega)$.
% \[
%   m^*(\partial (\phi(\Omega)))
%   \le (L\sqrt n + 1)^n \cdot m^*(\partial \Omega).
% \]
% Visto che $\Omega$ è PJ-misurabile, si ha $m^*(\partial \Omega)=0$,
% dunque anche $\partial \phi(\Omega)$ è misurabile 
% di misura nulla e, di conseguenza, $\phi(\Omega)$ è misurabile.
% Per continuità di $\phi$ abbiamo $\phi(\bar \Omega) \subset \overline{\phi(\Omega)}$
% e per compattezza di $\bar \Omega$ si ha $\overline{\phi(\bar \Omega)} = \phi(\bar \Omega)$,
% dunque $\overline{\phi(\Omega)} = \phi(\bar \Omega)$.
% Essendo $\bar \Omega$ compatto anche $\phi(\bar \Omega)$ è compatto, 
% dunque $\overline{\phi(\Omega)} \subset \phi(\bar \Omega)$
% Ovviamente anche $\phi(\bar \Omega) = \overline{\phi(\Omega)}$ è misurabile.

\emph{Passo 2 (lemma del cubo interno/esterno): 
se $\Omega = (0,1)^n$ è il cubo unitario  
ed esiste $\eps\in(0,\frac 1 4)$ tale che 
\[
  \abs{\phi(\vec x)-\vec x} \le \eps, \qquad \forall \vec x \in \bar \Omega,
\]
allora }
\[
  (1-2\eps)^n
  \le \abs{\phi(\Omega)}
  \le (1+2\eps)^n.
\]
Chiaramente 
\[
    \phi(\Omega) \subset [-\eps,1+\eps]^n
\]
perché ogni punto di $\Omega$ viene spostato al massimo di $\eps$. 
Quindi si ottiene, per monotonia della misura,
\[
    \abs{\phi(\Omega)} \le (1+2\eps)^n.
\]
Per ottener l'altra disuguaglianza  
\[
   \abs{\phi(\Omega)} \ge (1-2\eps)^n
\]
basterà dimostrare che vale anche 
\[
  \phi(\Omega) \supset (\eps,1-\eps)^n.
\]
Supponiamo per assurdo che esista $\vec y \in (\eps,1-\eps)^n \setminus \phi(\Omega)$.
Consideriamo il punto $\vec p=(\frac 12, \dots, \frac 12)$ centro 
del cubo $\Omega$ e sia $\vec q=\phi(\vec p)$ la sua immagine. 
Visto che $\abs{\vec q-\vec p} = \abs{\phi(\vec p)-\vec p} \le \eps$ 
e $\eps \le \frac 1 4$ possiamo garantire che
$\vec q \in (\eps,1-\eps)^n$.
Dunque l'intero segmento $\sigma = [\vec q,\vec y]$ è contenuto 
in $(\eps,1-\eps)^n$ essendo quest'ultimo un convesso.
Visto che $\vec q\in \phi(\Omega)$ mentre $\vec y \notin \phi(\Omega)$,
deve esistere un punto $\vec z\in \sigma\cap \partial \phi(\Omega)$
(esercizio~\ref{ex:punto-di-frontiera}).
Ma abbiamo già mostrato (Passo 1) che $\partial (\phi(\Omega)) \subset \phi(\partial \Omega)$
dunque $\vec z=\phi(\vec x)$ con $\vec x \in \partial \Omega$. 
Ma allora $\abs{\vec z-\vec x} = \abs{\phi(\vec x)- \vec x} \le \eps$ 
e dunque $\vec z\not \in (\eps,1-\eps)^n$, 
contraddicendo $\vec z \in \sigma \subset (\eps,1-\eps)^n$.

\emph{Passo 3: se $\Omega=(0,1)^n$ ed esiste $\eps>0$ tale che 
\[
  \Abs{ D\phi(\vec x) - I} \le \frac{\eps}{\sqrt n} \qquad \forall \vec x \in \bar \Omega,
\]
(dove $I$ è la matrice identità)
allora
\[
 (1-2\eps)^n \le \abs{\phi(\Omega)} \le (1+2\eps)^n.
\]
}
Se consideriamo la funzione $\psi(x) = \phi(x) - x$ osserviamo che $\psi(0)=\phi(0)$ e
\[
\Abs{D\psi(\vec x)} = \Abs{D\phi(\vec x) -I} \le \frac{\eps}{\sqrt n}.
\]
Applicando la disuguaglianza di Lagrange (teorema~\ref{th:lagrange}) si ottiene dunque:
\[
\abs{\phi(x) - \phi(0) - x} = \abs{\psi(x)-\psi(0)} \le \frac \eps {\sqrt n} \abs{x} \le \eps
\]
e la conclusione segue dal passo precedente applicato alla funzione $\phi(\vec x) - \phi(0)$.

\emph{Passo 4: supponiamo $\Omega=(0,1)^n$ e che esista una matrice invertibile $M$ e un $\eps>0$ tali che
\[
  \Abs{D\phi(\vec x) - M} \le \frac{\eps}{\Abs{M^{-1}} \sqrt n}, \qquad \forall \vec x \in \bar \Omega.
\]
allora
\begin{equation}\label{eq:475681}
(1-2\eps)^n \le \frac{\abs{\phi(\Omega)}}{\abs{\det M}} \le (1+2\eps)^n.
\end{equation}
}

Consideriamo la funzione $\psi(\vec x) = M^{-1} \phi(\vec x)$ per la quale si ha  
\[
  \Abs{D\psi(\vec x)-I} 
  = \Abs{M^{-1} (D\phi(\vec x)-M)} 
  \le \Abs{M^{-1}} \cdot \Abs{D\phi(\vec x)-M}
  \le \frac{\eps}{\sqrt n}.
\]
Applicando il passo precedente alla funzione $\psi$ otteniamo 
\[
(1-2\eps)^n \le \abs{\psi(\Omega)} \le (1+2\eps)^n.
\]
Ricordando il Teorema \ref{th:geometria_determinante} si ha
\[
  \abs{\psi(\Omega)} = \abs{M^{-1}(\phi(\Omega))}
  = \abs{\det M^{-1}} \cdot \abs{\phi(\Omega)}
  = \frac{\abs{\phi(\Omega)}}{\abs{\det M}}
\]
da cui si ottiene~\eqref{eq:475681}.

\emph{Passo 5: se $Q^N_{\vec k}\subset \bar \Omega$ è un cubetto di lato $2^{-N}$
e se esistono una matrice invertibile $M$ ed $\eps>0$ tali che per ogni $\vec x \in Q^N_{\vec k}$
\[
  \Abs{D\phi(x)-M} \le \frac{\eps}{\Abs{M^{-1}} \sqrt n},
\]
allora
\[
(1-2\eps)^n \le \frac{\abs{\phi(Q^N_{\vec k})}}{2^{-nN}\abs{\det M}} \le (1+2\eps)^n.
\]
}
Sia $\vec x^N_{\vec k} = 2^{-N}\vec k$ il vertice del cubetto $Q^N_{\vec k}$.
Basta considerare la funzione $\psi\colon Q\to \RR^n$ 
definita da $\psi(\vec x) = 2^N(\phi(\vec x^N_{\vec k} + 2^{-N}\vec x)-\phi(\vec x^N_{\vec k}))$ 
e osservare che
\begin{gather*}
  D\psi(\vec x)
  = D\phi(\vec x^N_{\vec k} + 2^{-N}\vec x), \qquad \psi(0)=0\\
%  \int_{Q^N_{\vec k}} \abs{\det M}\, d\vec x
%  = 2^{nN}\cdot\abs{\det M}, \qquad
  \abs{\psi(Q)} = 2^{nN} \cdot \abs{\phi(Q^N_{\vec k})}
\end{gather*}
cosicché si può applicare il passo precedente alla funzione $\psi$
per ottenere quanto affermato.

\emph{Passo 6: conclusione.}
Essendo $\bar \Omega$ compatto e $D\phi$ continua, 
anche $\abs{\det D\phi}$ è continua e, per il teorema di Weierstrass,
esistono $L,\mu<+\infty$ tali che per ogni $\vec x \in \bar \Omega$ si ha
\[
  \abs{D\phi(\vec x)} \le L,
  \qquad
  \abs{\det D\phi(\vec x)} \le \mu.
\]

Fissiamo $\eps>0$.
Visto che $\Omega$ è misurabile esiste $M\in \NN$ abbastanza grande in modo che 
\[
  m^M(\bar \Omega)-m_M(\Omega) < \frac{\eps}{(\sqrt n L +1)^n + \mu}.
\]
Consideriamo il compatto $\Omega_M$ che è l'unione di tutti i cubetti $Q^M_{\vec k}$
contenuti interamente in $\Omega$.
Per il Lemma~\ref{lm:misura-immagine-lipschitz} si ha 
\[
 m^*(\phi(\Omega_M)) 
 \le (L\sqrt n + 1)^n \cdot \abs{\Omega_M} 
 \le (L\sqrt n + 1)^n \cdot (m^M(\bar \Omega) - m_M(\Omega))
 \le \eps.
\]
D'altra parte anche 
\[
\int_{\Omega\setminus \Omega_M} \abs{\det D\phi}
\le \mu \abs{\Omega\setminus \Omega_M} \le \eps
\]
dunque per concludere la dimostrazione basterà mostrare che 
\begin{equation}\label{eq:39254}
  \abs{\phi(\Omega_M)} = \int_{\Omega_M} \abs{\det D\phi} 
\end{equation}
in quanto 
\[
   \abs{\abs{\phi(\Omega)} - \int_{\Omega} \abs{\det D\phi}}
   \le \abs{\abs{\phi(\Omega_M)} - \int_{\Omega_M} \abs{\det D\phi}} + 2\eps
\]
e il risultato seguirebbe immediatamente facendo tendere $\eps\to 0^+$.

Per dimostrare~\eqref{eq:39254} andremo a suddividere i cubetti di lato $2^{-M}$
che formano $\Omega_M$ in cubetti ancora più piccoli.
Fissato $\eps>0$,
per il teorema di Heine-Cantor  
esiste $N\in \NN$ tale che 
per ogni $\vec x, \vec x' \in Q^N_{\vec k}\cap \Omega$ si ha
\[
  \Abs{D\phi(\vec x) - D\phi(\vec x')} \le \frac{\eps}{\lambda \sqrt n},
  \qquad
  \abs{\det D\phi(\vec x) - \det D\phi(\vec x')} \le \eps.
\]
Per ogni cubetto $Q^N_{\vec k} \subset \Omega_m$ sia $\vec x^N_{\vec k} \defeq 2^{-N}\vec k$ 
il vertice del cubetto e sia $M^N_{\vec k} \defeq D\phi(\vec x^N_{\vec k})$.
Allora sono soddisfatte le condizioni per applicare il passo precedente, e si 
ottiene 
\[
(1-2\eps)^n \le \frac{\abs{\phi(Q^N_{\vec k})}}{2^{-Nn}\abs{\det M^N_{\vec k}}} \le (1+2\eps)^n.
\]
Osservando che 
\[
 0 < 1-(1-2\eps)^n < (1+2\eps)^n - 1 = 2n\eps + o(\eps) \qquad \text{per $\eps\to 0^+$},
\]
e ricordando che
$
%\int_{Q^N_{\vec k}} \abs{\det M^N_{\vec k}}
= 2^{nN} \abs{\det M^N_{\vec k}} \le 2^{nN} \mu
$,
possiamo riscrivere le disuguaglianze precedenti nella forma
\[
\abs{\abs{\phi(Q^N_{\vec k})} - 2^{-Nn} \abs{\det M^N_{\vec k}}}
\le (2n\eps + o(\eps))\cdot 2^{-Nm} \mu.
\]
Inoltre, visto che
\[
 \abs{\int_{Q^N_{\vec k}} \abs{\det D\phi(\vec x)}\, dx - 
 \int_{Q^N_{\vec k}} \abs{\det M^N_{\vec k}}\, d\vec x}
 \le \int_{Q^N_{\vec k}} \abs{\det D\phi(\vec x) - \det M^N_{\vec k}}\, d\vec x
 \le 2^{-Nn} \eps
\]
otteniamo infine 
\[
\abs{\abs{\phi(Q^N_{\vec k})} - \int_{Q^N_{\vec k}} \abs{\det D\phi}}
= \abs{\abs{\phi(Q^N_{\vec k})} - 2^{-Nn} \abs{\det D\phi}}
\le (2n\eps + o(\eps))\cdot 2^{-Nm} \mu + 2^{-Nn} \eps.
\]
Visto che $\phi$ è iniettiva su $\Omega_M\subset \Omega$, 
le parti interne degli insiemi $\phi(Q^N_{\vec k})$ sono a due a due disgiunte 
dunque, sommando su tutti i cubetti $Q^N_{\vec k}\subset K_0$,
si ottiene
\begin{align*}
\abs{\abs{\phi(\Omega_M)} - \int_{\Omega_M} \abs{\det D\phi}}
&=\abs{\sum_{Q^N_{\vec k}\subset \Omega_M}
  \abs{\phi(Q^N_{\vec k})} - \sum_{Q^N_{\vec k}\subset \Omega_M}\int_{Q^N_{\vec k}} \abs{\det D\phi}} \\
&\le \sum_{Q^N_{\vec k}\subset \Omega_M} 
  \abs{\abs{\phi(Q^N_{\vec k})} - \int_{Q^N_{\vec k}} \abs{\det D\phi}}\\
  &\le (2n\eps + o(\eps))\cdot \mu + \eps.
\end{align*}
Facendo tendere $\eps\to 0^+$ si ottiene~\eqref{eq:39254}, concludendo la dimostrazione.
\end{proof}


\begin{theorem}[cambio di variabili]
  \label{th:cambio-variabili}
Sia $\Omega \subset \RR^n$ un aperto convesso $PJ$-misurabile, 
$\phi\colon \bar \Omega \to \RR^n$
una funzione di classe $C^1$, 
$\phi$ iniettiva se ristretta ad $\Omega$,
$\det\phi(x)\neq 0$ per ogni $x\in \Omega$.
Allora $\phi(\Omega)$ è misurabile e 
se $f\colon \phi(\Omega)\to\RR$ è una funzione localmente integrabile,
si ha 
\[
    \int_\Omega f(\phi(\vec x))\cdot \abs{\det D \phi(\vec x)}\, d \vec x
    =
    \int_{\phi(\Omega)} f(\vec y) \, d \vec y.
\]
\end{theorem}

\begin{proof}
  \emph{Case 1: supponiamo che $\Omega$ sia limitato e che $f$ sia limitata.}
Siano $L,M,\sigma$ costanti positive tali che 
per ogni $\vec x\in \bar \Omega$ si abbia
\[
  \abs{f(x)} \le M, \qquad
  \Abs{D\phi(\vec x)} \le L.
\]
Essendo $\Omega$ convesso, possiamo applicare il 
teorema~\ref{th:lagrange} di Lagrange 
per ottenere la condizione di Lipschitz:
\[
  \abs{\phi(\vec x) - \phi(\vec y)} \le \Abs{D\phi(\vec z)}L.
\]
Per il Lemma~\ref{lm:misura-immagine-lipschitz} troviamo 
dunque che $\phi(\Omega)$ è misurabile.
Dato $N\in \NN$ consideriamo l'insieme compatto
\[
  \Omega^N \defeq \bigcup_{Q^N_{\vec k}\subset \Omega} Q^N_{\vec k}
\]
formato dai cubetti di lato $2^{-N}$ contenuti in $\Omega$.
Posto $F^N_{\vec k} \defeq \sup \phi(Q^N_{\vec k})\in \RR$,
si ha, usando la definizione di integrale Riemann,
e il teorema~\ref{th:formula-area}:
\begin{align*}
  S^N(f) 
    &= \sum_{Q^N_{\vec k}\subset \Omega} F^N_{\vec k} \cdot \abs{Q^N_{\vec k}}
       + \sum_{Q^N_{\vec k}\cap \Omega  \emptyset} F^N_{\vec k} \cdot \abs{Q^N_{\vec k}\cap \Omega}\\
    &= \sum_{\vec k \in \ZZ^n} F^N_{\vec k} \cdot \abs{\phi(\phi^{-1}(Q^N_{\vec k}))}\\
    &= \sum_{\vec k \in \ZZ^n} F^N_{\vec k} \cdot 
      \int_{\phi^{-1}(Q^N_{\vec k})} \abs{\det D\phi(\vec x)}\, d\vec x
    = \sum_{\vec k \in \ZZ^n} 
      \int_{\phi^{-1}(Q^N_{\vec k})} F^N_{\vec k} \abs{\det D\phi(\vec x)}\, d\vec x\\
    &\ge \int_{\Omega} f(\phi(\vec x)) \abs{\det D\phi(\vec x)}\, d\vec x.
\end{align*}


*************
  \emph{Caso 1: supponiamo $E$ compatto.}
Allora $\Abs{D\phi}$ ha massimo $L$ su $E$, e $f$ ha massimo $M$ 
su $\phi(E)$. 

Visto che $D\phi$ è continua su $E$, è anche uniformemente continua e 
usando il teorema di Lagrange si può mostrare che per ogni $\eps>0$ 
esiste $\delta>$ tale che
\[
 \abs{\vec x - \vec y}<\delta 
 \implies 
 \abs{\phi(\vec x)-\phi(\vec y)-D\phi(\vec y)(\vec x - \vec y)}
 \le \eps (\vec x - \vec y).
\]
Sia dato $\eps\in(0,1)$.
Sia $N\in \NN$ abbastanza grande in modo che, posto
\begin{align*}
  K&=\ENCLOSE{\vec k\in \ZZ^n\colon Q^N_{\vec k}\subset E},\\
  K'&=\ENCLOSE{\vec k\in \ZZ^n\setminus K \colon Q^N_{\vec k}\cap E\neq \emptyset}.
\end{align*}
e posto, per $\vec k \in K$,dove 
\[
  \deg(\phi,\vec y) = \sum_{\phi(\vec x)=\vec y} \sgn \det \phi(\vec x).
\]

\begin{gather*}
  \vec x^N_{\vec k} = 2^{-N}\vec k,\\
  f^N_{\vec k} = f(\phi(\vec x^N_{\vec k})),\\ 
  \Phi^N_{\vec k} = D\phi(\vec x^N_{\vec k}), 
\end{gather*}
si abbia
\begin{gather*}
  M L^d (S^N(1_E)-S_N(1_E)) < \eps,\\
  \forall \vec k\in K \colon \forall \vec x \in Q^N_{\vec k}\colon
    \abs{f(\phi(\vec x))-f^N_{\vec k}} < \eps,\\
  \forall \vec k\in K \colon \forall \vec x \in Q^N_{\vec k}\colon
    \abs{\phi(\vec x) - \phi(\vec x^N_{\vec k})-\Phi^N_{\vec k}[\vec x-\vec x^N_{\vec k}]} < \eps 2^{-N}.
\end{gather*}

Per il Lemma, sappiamo che se $\vec k \in K$ allora $\phi(Q^N_{\vec k})$ è PJ-misurabile 
e si ha 
\[
 \abs{\phi(Q^N_{\vec k})} 
 \le (L\sqrt n)^d 2^{-nN}
\]

\begin{align*}
  \abs{\int_{\phi(E)} f(\vec y)\, d\vec y 
  - \sum_{\vec k \in K} \int_{\phi(Q^N_{\vec k})} f(\vec y)\, d\vec y}
  &\le M \cdot \#K' \cdot (\sqrt n K 2^{-N})^n \\
  &\le M (S^N(1_E)-S_N(1_E)) (\sqrt n K)^n
  \le \eps 
\end{align*}

\begin{align*}
\sum_{\vec k \in K}\int_{\phi(Q^N_{\vec k})} \abs{f(\vec y) - f^N_{\vec k}}\, d\vec y
&\le  \abs{\phi(E)} \eps 
\end{align*}

Ora notiamo che siccome per $\vec x \in Q^N_{\vec k}$ si ha 
\[
  \abs{
    \phi(\vec x) - \phi(\vec x^N_{\vec k}) 
    - \Phi^N_{\vec k}(\vec x - \vec x^N_{\vec k})} \le \eps 2^{-N}
\]
risulta 
\begin{align*}
 \phi(Q^N_{\vec k}) - \phi(\vec x^N_{\vec k}) \subset \Phi(Q^N_{\vec k})+B_{\eps 2^{-N}}
\end{align*}

\begin{align*}
\sum_{\vec k \in K}
  \int_{\phi(Q^N_{\vec k})} f^N_{\vec k}\, d\vec y 
= \sum_{\vec k in K} f^N_{\vec k} \abs{\phi(Q^N_{\vec k})}
\end{align*}

***

Per Taylor per ogni $N\in \NN$ e $\vec k\in \ZZ^n$,
posto ${\vec x}^N_{\vec k} = 2^{-N}\vec k$ e $\Phi^N_{\vec k}=D\phi({\vec x}^N_{\vec k})$,
si ha 
\begin{align*}
  \phi(\vec x) &= \phi({\vec x}^N_{\vec k}) + \Phi^N_{\vec k} (\vec x - {\vec x}^N_{\vec k}) + \eps 2^{-Nn}\\
  f(\phi(\vec x)) &= F^N_{\vec k} + \eps\\
  \abs{f(\vec x)} & \le M\\
  \abs{\det D\phi(\vec x)} & \le M\\
\end{align*}
cosicché
\begin{align}
  \abs{
    F^N_{\vec k}  2^{-Nn} \det \Phi^N_{\vec k}
    - \int_{Q^N_{\vec k}} f(\phi(\vec x)) \det D\phi(\vec x)\, dx
  }
  &\le \int_{Q^N_{\vec k}}  \Enclose{M \eps 2^{-nN} + \eps M}\, dx \\
  &\le 2M\eps 2^{-nN}. 
\end{align}
ma 
\begin{align*}
    F^N_{\vec k} 2^{-nN} \det \Phi^N_{\vec k} 
    &= F^N_{\vec k}\abs{\phi(Q^N_{\vec k})}.
\end{align*}
e
\begin{align*}
  \int_{\phi(Q^N_{\vec k})} f(\vec y) \deg(\phi,\vec y)\, dy 
  &=
  \sum
\end{align*}
\end{proof}