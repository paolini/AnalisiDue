\chapter{integrali multipli}

\section{integrale di Riemann}
Se $x,y\in \RR$ definiamo:
\[
    x \vee y = \max\ENCLOSE{x,y},
    \quad
    x \wedge y = \min\ENCLOSE{x,y},
    \quad
    x^+ = x \vee 0,
    \quad
    x^- = (-x)^+ = -(x\wedge 0).
\]
Si ha
\[
    x^+,x^-\ge 0,
    \quad
    x = x^+ - x^-,
    \quad 
    \abs{x} = x^+ + x^-,
    \quad
    x^+ = \frac{\abs{x}+x}{2},
    \quad
    x^- = \frac{\abs{x}-x}{2}.
\]

Sia $A\subset \RR^n$ e $f\colon A \to \RR$. 
Definiamo il \emph{supporto} di $f$ come 
\[
 \spt f 
    = \ENCLOSE{\vec x \in A \colon f(\vec x) \neq 0}
    = f^{-1}(\RR\setminus\ENCLOSE{0}).
\]

Dato $A\subset \RR^n$ definiamo la \emph{funzione caratteristica}
di $A$ come $1_A\colon \RR^n \to \RR$,
\[
  1_A(\vec x) = \begin{cases}
    1 & \text{se $\vec x \in A$,}\\
    0 & \text{altrimenti.}
  \end{cases}
\]

Per $\vec k \in \ZZ^d$ e $N\in \NN$ definiamo 
il cubetto 
\[
  Q^N_{\vec k} = \frac{\vec k + [0,1]^d}{2^N}.
\]
Si tratta di un cubetto di lato $2^{-N}$ con gli spigoli paralleli 
agli assi cartesiani e con un vertice nel punto $2^{-N} \vec k$.
Il volume di $Q^N_{\vec k}$ è quindi $2^{-nN}$.
Useremo questi cubetti per misurare gli insiemi.

\begin{definition}[integrale di Riemann]
  \label{def:integrale_riemann}
Sia $f\colon \RR^n \to \RR$ una funzione limitata, 
con supporto limitato.
Definiamo le somme superiori e inferiori sulla $N$-esima suddivisione in cubi:
\begin{align*}
    S^N(f) &= 2^{-nN}\sum_{\vec k\in \ZZ^n} \sup f(Q^N_{\vec k}), \\
    S_N(f) &= 2^{-nN}\sum_{\vec k\in \ZZ^n} \inf f(Q^N_{\vec k}),
\end{align*}
e quindi, osservando che $N\mapsto S^N(f)$ è decrescente 
mentre $N\mapsto S_N(f)$ è crescente, definiamo
l'integrale superiore e inferiore:
\begin{align*}
    S^*(f) &= \lim_{N\to +\infty} S^N(f) \in \RR, \\
    S_*(f) &= \lim_{N\to +\infty} S_N(f) \in \RR.
\end{align*}
Se $S^*(f)=S_*(f)$ diremo che $f$ è Riemann integrabile (o R-integrabile)
e scriveremo
\[
  \int f = S^*(f) = S_*(f).
\]
\end{definition}

\begin{theorem}[monotonia dell'integrale]
Siano $f,g\colon \RR^n\to\RR$ due funzioni limitate con supporto limitato.
Se $f$ e $g$ sono $R$-integrabili si ha:
\[
  f \le g \implies \int_A f \le \int_A g.
\]
\end{theorem}
%
\begin{proof}
Se $f\le g$ per ogni $N\in \NN$ e ogni $\vec k \in \ZZ^n$ si ha 
\[
    \sup f(Q^N_{\vec k}) \le \sup g(Q^N_{\vec k}) 
\]
da cui 
\[
    S^N(f) \le S^N(g)
\]
e passando al limite per $N\to +\infty$,
\[
  S^*(f) \le S^*(g)
\]
che è la tesi se $f$ e $g$ sono R-integrabili.
\end{proof}

\begin{theorem}[linearità dell'integrale]
Siano $f,g\colon \RR^n \to \RR$ funzioni limitate con supporto limitato.
Se $f,g$ sono R-integrabili allora $f+g$ è R-integrabile e si ha 
\[
    \int (f+g) = \int f + \int g.
\]
Se $f$ è R-integrabile e $\lambda\in \RR$ anche $\lambda f$ è R-integrabile 
e si ha 
\[
    \int \lambda f = \lambda \int f.
\]
\end{theorem}
\begin{proof}
Per ogni $\vec k\in\ZZ^n$ e ogni $N\in \NN$
si ha
\begin{align*}
  \sup\, (f+g)(Q^N_{\vec k}) &\le \sup f(Q^N_{\vec k}) + \sup g(Q^N_{\vec k})\\
  \inf\, (f+g)(Q^N_{\vec k}) &\ge \inf f(Q^N_{\vec k}) + \inf g(Q^N_{\vec k}).
\end{align*}
Dunque si ottiene
\[
    S^N(f+g) \le S^N(f) + S^N(g),  \qquad
    S_N(f+g) \ge S_N(f) + S_N(g), 
\]
e passando al limite per $N\to +\infty$ 
\[
   S_*(f) + S_*(g) \le S_*(f+g) \le S^*(f+g) \le S^*(f) + S^*(g).
\]
Se $f$ e $g$ sono R-integrabili si ottiene dunque che anche $f+g$ 
è R-integrabile e inoltre $\int (f+g) = \int f + \int g$.

Se $\lambda \ge 0$ si ha 
\[
   S^N(\lambda f) = \lambda S^N(f), \qquad S_N(\lambda f) = \lambda S_N(f)
\]
e quindi se $f$ è R-integrabile si conclude 
che $\lambda f$ è R-integrabile e $\int \lambda f = \lambda \int f$.

Se $\lambda = -1$ si osserva che per ogni $N\in \NN$, $\vec k\in \ZZ^n$, 
si ha 
\[
   \sup (-f)(Q^N_{\vec k}) = -\inf f(Q^N_{\vec k}),
   \qquad 
   \inf (-f)(Q^N_{\vec k}) = -\sup f(Q^N_{\vec k}),
\]
da cui 
\[
    S^N(-f) = -S_N(f), \qquad S_N(-f) = -S^N(f)
\]
e quindi, per $N\to +\infty$ si ottiene 
\[
  S^*(-f) = -S_*(f), \qquad S_*(-f) = - ^S*(f).
\]
Se $f$ è R-integrabile deduciamo che anche $-f$ lo è e si ha 
$\int (-f) = -\int f$.

Combinando i casi precedenti si ottiene il caso generale $\lambda \in \RR$.
\end{proof}

\begin{theorem}[proprietà di reticolo]
Sia $f\colon \RR^n \to \RR$ una funzione limitata con supporto limitato.
Se $f$ è R-integrabile allora anche $\abs{f}$, $f^+$ e $f^-$ sono 
R-integrabili e si ha 
\[
  \abs{\int f} \le \int \abs f = \int f^+ + \int f^-, 
  \qquad 
 \int f = \int f^+ - \int f^-.
\]
\end{theorem}
\begin{proof}
Si osserva che per ogni $N\in \NN$, $\vec k \in \ZZ^n$, si ha 
\[
  \sup \abs{f}(Q^N_{\vec k}) - \inf \abs{f}(Q^N_{\vec k})
  \le 
  \sup f(Q^N_{\vec k}) - \inf f(Q^N_{\vec k})
\]
e quindi 
\[
   S^N(\abs f) - S_N(\abs f) \le S^N(f) - S_N(f)
\] 
da cui, passando al limite per $N\to +\infty$,
\[
  S^*(\abs f) - S_*(\abs f) \le S^*(f) - S_*(f)
\]
per cui se $f$ è R-integrabile anche $\abs{f}$ lo è.

Osserviamo ora che $f^+=\frac{\abs f + f}{2}$ e $f^-=\frac{\abs f - f}{2}$ 
e dunque per linearità se $f$ è R-integrabile avendo già dimostrato 
che anche $\abs f$ è R-integrabile, si ottiene che pure $f^+
$ e $f^-$ sono R-integrabili ed essendo $f=f^+-f^-$, $\abs f = f^+ + f^-$ si ottiene 
\[
  \int f = \int f^+ - \int f^-, \qquad 
  \int \abs f = \int f^+ + \int f^-.
\]
Per la convessità del valore assoluto si ottiene infine
\[
 \abs{\int f} = \abs{\int f^+ - \int f^-} 
 \le \abs {\int f^+} + \abs{\int f^-}
 = \int f^+ + \int f^- = \int \abs f.
\]
\end{proof}

\begin{definition}[integrale di Riemann generalizzato]
Se $f\colon \RR^n \to [0,+\infty]$ è una funzione non negativa, per ogni $R>0$ 
possiamo definire il troncamento $f_R$ 
\[
  f_R = (f \cdot 1_{[-R,R]^n}) \vee R.
\]
La funzione $f_R$ è limitata ed ha supporto limitato.
Se $f_R$ è Riemann-integrabile per ogni $R>0$ diremo che $f$ è localmente R-integrabile 
e visto che $R\mapsto f_R$ è crescente possiamo definire
\[
 \int f = \lim_{R\to+\infty} \int f_R \in [0,+\infty].
\]
Se $\int f<+\infty$ diremo che $f$ è R-integrabile.

Se $f\colon \RR^n\to \RR$ possiamo scrivere $f= f^+-f^-$ con $f^+,f^-$ funzioni non negative.
Se $f^+$ e $f^-$ sono localmente R-integrabili e se 
$\int f^+$ e $\int f^-$ non sono entrabi infiniti possiamo definire 
\[
  \int f = \int f^+ - \int f^- \in [-\infty,+\infty].
\]
Se $\int f$ è finito diremo che $f$ è $R$-integrabile.
\end{definition}

\begin{definition}
Dato $E\subset \RR^n$ diremo che $E$ è \emph{Peano-Jordan misurabile} 
(o PJ-misurabile) se la funzione caratteristica $1_E$ è localmente 
R-integrabile, in tal caso definiamo la misura di $E$ come 
\[
\abs{E} = \int 1_E \in [0,+\infty].
\] 

Se $A\subset \RR^n$ è PJ-misurabile e $f\colon A \to \RR$
è una funzione qualunque, possiamo considerare l'estensione 
$\tilde f\colon \RR^n \to \RR$ definita da 
\[
 \tilde f(\vec x) = \begin{cases}
    f(\vec x) & \text{se $x\in A$,}\\
    0 & \text{altrimenti.}
 \end{cases}
\]
Diremo che $f$ è localmente R-integrabile in $A$ se $\tilde f$ 
è localmente R-integrabile. 
Scriveremo:
\[
 \int_A f = \int \tilde f
\]
se $\int \tilde f$ è definito.
Se inoltre $\int_A f$ è finito diremo che $f$ è R-integrabile
su $A$.
\end{definition}

\begin{remark}[caso $n=1$]
Nel caso $n=1$ se $f\colon [a,b]\to \RR$ è una funzione limitata,
si osserva che la suddivisione in cubetti di lato $2^{-N}$ induce una partizione 
su $[a,b]$ e la definizione di Riemann integrabilità data sopra coincide con la 
usuale Riemann integrabilità definita tramite suddivisioni arbitrarie.

Nel caso $f\colon(a,b)\to \RR$ con $(a,b)$ non limitato oppure $f$ non limitata,
la definizione data sopra corrisponde invece alla \emph{assoluta convergenza} 
dell'integrale improprio.
\end{remark}

\begin{theorem}[algebra dei misurabili]
Se $E,F\subset \RR^n$ sono PJ-misurabili allora anche $\bar E$, $\partial E$, 
$E\setminus F$, $E\cap F$ ed $E\cup F$ sono PJ-misurabili.
Inoltre si ha 
\[
    \abs{E\cup F} + \abs{E\cap F} = \abs{E} + \abs{F},
    \qquad
    \abs{\partial E} = 0.
\]
\end{theorem}
\begin{proof}
Per la misurabilità di $E\setminus F$ e di $E\cap F$, basta osservare che 
\begin{align*}
  1_{E\setminus F} &= (1_E - 1_F)^+\\
  1_{E\cap F} &= 1_E - 1_{E\setminus F}\\
  1_{E\cup F} &= 1_F + 1_{E\setminus F}
\end{align*}
e usare le proprietà di reticolo.
Osservando poi che 
\[
 1_E + 1_F = 1_{E\cup F} + 1_{E\cap F}
\]
si ottiene la formula
\[
\abs{E} + \abs{F} = \abs{E\cup F} + \abs{E\cap F}.
\]

Supponiamo ora che $E$ sia limitato
e dimostriamo che $\partial E$ ha misura nulla.
Bisogna osservare 
che $\sup 1_{\partial E}(Q^N_{\vec k}) = 1$ 
significa che $\partial E\cap Q^N_{\vec k} \neq \emptyset$.
Ma allora c'è un punto $\vec x\in Q^N_{\vec k}$ tale che ogni intorno 
di $\vec x$ contiene sia punti di $E$ che punti di $\RR^n\setminus E$.
Se il punto $\vec x$ è interno a $Q^N_{\vec k}$ si ha
\[
  \sup 1_{\partial E}(Q^N_{\vec k}) > \inf 1_{\partial E}(Q^N_{\vec k}).
\]
Se ciò non fosse vero significa che $\vec x$ sta sulla frontiera del cubetto 
$Q^N_{\vec k}$ e la disuguaglianza precedente risulta valida 
per il cubetto $Q^N_{\vec k'}$ adiacente a $Q^N_{\vec k}$.
Dunque si deve avere 
\[
  S^N(1_{\partial E})\le 3^d (S^N(1_E)-S_N(1_E))
\]
(dove $3^d-1$ è il numero di cubetti adiacenti ad un cubetto fissato)
e per $N\to +\infty$ si ottiene 
\[
  S^*(1_{\partial E}) \le 2(S^*(1_E)-S_*(1_E)) = 0.
\]
Dunque $E$ è misurabile e $\abs{E} = 0$.

Se $E$ non è limitato si ripete l'osservazione precedente alle intersezioni 
$E\cap [-R,R]^n$ per ottenere lo stesso risultato.

Visto che $\bar A = A \cup \partial A$, se $A$ è misurabile anche $\bar A$ è 
misurabile e si ha $\abs{\bar A} = \abs{A\setminus \partial A} + \abs{\partial A}
= \abs{A \setminus \partial A} \le \abs{A} \le \abs{\bar A}$,
da cui $\abs{\bar A}=\abs{A}$.
\end{proof}

\begin{theorem}[significato geometrico dell'integrale]
Se $f\colon A\subset \RR^n \to \RR$ definiamo la parte positiva dell'epigrafico di $f$
come:
\[
  E_f = \ENCLOSE{(\vec x,y)\in A\times \RR \colon 0 \le y \le f(\vec x)} \subset \RR^{n+1}.
\]
Se $A\subset \RR^n$ è PJ-misurabile ed $f$ è localmente integrabile 
allora $E_{f^+}$ ed $E_{f^-}$ sono localmente misurabili e se non
hanno entrambi misura infinita si ha 
\[
  \int_A f = \abs{E_{f^+}} - \abs{E_{f^-}}.
\]
\end{theorem}

\begin{proof}
Supponiamo $f\ge 0$, $f$ limitata, $A$ limitato PJ-misurabile.
In tal caso si può osservare che 
per ogni cubetto $n$-dimensionale che tocca la base $A$, 
i cubetti $(n+1)$-dimensionali che coprono il grafico di $f$ 
sono al massimo due in più rispetto all'altezza della colonna, 
ovvero del $\sup$ di $f$ 
sul cubetto base. Per i cubetti interni si ha invece l'opposto. Cioè:
\begin{align*}
    0 & \le S^N(1_{E_{f^+}}) - S^N(f) \le 2\cdot 2^{-N} S^N(1_A),\\
    0 & \le S_N(f) - S_N(1_{E_{f^+}}) \le 2\cdot 2^{-N} S^N(1_A)
\end{align*}
e, per $N\to +\infty$, si ottiene 
\begin{align*}
  0 &\le S^*(1_{E_{f^+}}) - S^*(f) \le 0\cdot S^*(1_A), \\
  0 &\le S_*(f) - S_*(1_{E_{f^+}}) \le 0\cdot S^*(1_A), 
\end{align*}
ovvero $\abs{E_{f^+}}=\int f$, come volevamo dimostrare.

Gli altri casi vengono di conseguenza separando parte positiva e parte negativa 
e passando al limite nei troncamenti.
\end{proof}

\begin{theorem}[integrabilità funzioni continue]
  \label{th:continua_integrabile}
Sia $A\subset \RR^n$ un insieme chiuso PJ-misurabile e $f\colon A \to \RR$ una funzione 
continua. Allora $f$ è localmente R-integrabile. 
Se inoltre $\abs{A}<+\infty$ e $f$ 
è limitata, allora $f$ è R-integrabile.
\end{theorem}
%
\begin{proof}
\emph{Passo 1: supponiamo $A$ compatto RJ-misurabile, 
$f$ continua, limitata, non negativa.}
Per il teorema di Weierstrass esiste $M>0$ tale che $0\le f(x) \le M$ per 
ogni $\vec x \in A$.
Fissiamo $\eps>0$.
Per il teorema di Heine-Cantor esiste $\delta>0$ tale 
che se $\vec x_1,\vec x_2\in A$ e $\abs{\vec x_1 -\vec x_2}<\delta$ 
allora $\abs{f(\vec x_1)-f(\vec x_2)}< \eps$. 
Scegliamo $N\in \NN$ abbastanza grande in modo che si abbia $2^{-N}\sqrt n < \delta$
e in modo che $S^N(1_{\partial A})<\eps$.
Allora si ha
\[
  S^N(f) - S_N(f) 
  = 2^{-nN}\sum_{\vec k} (\max f(Q^N_{\vec k})-\min f(Q^N_{\vec k})) 
  \le 2^{-nN}\eps S^N(1_A).
\]
Visto che $f$ è limitata e $A$ è limitato, $S^N(f)$ è limitata in $N$ 
e facendo tendere $N\to+\infty$ si ottiene 
dunque $S^*(f)-S_*(f) \le 0$ da cui $S^*(f)=S_*(f)$ come volevamo dimostrare.

\emph{Passo 2: supponiamo $A$ chius PJ-misurabile, $f$ continua, non negativa.} 
Si applica il passo precedente alla troncata $f_R$.

\emph{Passo 3: caso generale.}
Si applica il passo precedente alle parti 
positiva e negativa $f^+$, $f^-$ di $f$.

\emph{Passo 4: supponiamo $\abs{A}<+\infty$ e $f$ limitata.}  
In tal caso esiste $M$ tale che $\abs{f(\vec x)}\le M$ per 
ogni $x\in A$. 
Per la monotonia dell'integrale si ottiene 
\[
  \abs{\int_A f} 
  \le \int_A \abs{f}
  \le \int_A M = M \cdot \abs{A} < +\infty.  
\]
\end{proof}

Volendo rendere esplicita la variabile di integrazione, potremo 
scrivere 
\[
  \int_A f(x)\, dx = \int_A f.
\]

\begin{theorem}[integrabilità del prodotto]
Sia $f\colon \RR^n\to \RR$ una funzione continua e $g\colon \RR^n\to \RR$ 
una funzione R-integrabile. Allora $f\cdot g$ è R-integrabile.
\end{theorem}
\begin{proof}
\emph{Passo 1. Supponiamo $f,g\ge 0$, $f,g$ limitate con supporto limitato.}
Sia $M_f=\sup f$, $M_g=\sup g$.
Definiamo, per $A\subset \RR$,
\[
  \osc A = \sup A - \inf A.
\]
Fissiamo $\eps>0$.
Essendo $f$ uniformemente continua, se $N$ è abbastanza grande 
per ogni $\vec k\in \ZZ^n$ si ha $\osc f(Q^N_{\vec k})\le \eps$.
Dunque:
\begin{align*}
S^N(f\cdot g) - S_N(f\cdot g)
&= \sum_{\vec k\in \ZZ^n} \osc (f\cdot g)(Q^N_{\vec k})\\
&\le \sum_{\vec k}\Enclose{\sup f(Q^N_{\vec k}) \cdot \sup g(Q^N_{\vec k})
  -\inf f(Q^N_{\vec k})\cdot \inf g(Q^N_{\vec k})} \\
&= \sum_{\vec k}\Enclose{\sup f(Q^N_{\vec k}) \osc g(Q^N_{\vec k}) 
  + \osc f(Q^N_{\vec k})\cdot \inf g(Q^N_{\vec k})}\\
&\le M_f (S^N(g)-S_N(g)) + \eps M_g.
\end{align*}
Facendo tendere $N\to +\infty$ si ottiene 
\[
S^*(f\cdot g)- S_*(f\cdot g) 
\le M_f (S^*(g)-S_*(g)) + \eps M_g = \eps M_g
\]
essendo $g$ R-integrabile.
Facendo tendere $\eps\to 0$ si ottiene infine $S^*(f\cdot g)=S_*(f\cdot g)$ 
che è quanto volevamo dimostrare.
\end{proof}

\begin{theorem}[Fubini-Tonelli, formule di riduzione]
Sia $f\colon \RR^n\times \RR^m \to \RR$ R-integrabile
tale che per ogni $\vec x\in \RR^n$ la funzione $\vec y \mapsto f(\vec x,\vec y)$ 
risulti anch'essa R-integrabile su $\RR^m$.
Allora 
\[
  \int_{\RR^{n+m}} f = \int_{\RR^n} \Enclose{\int_{\RR^m} f(\vec x, \vec y)\, d\vec y}\, d\vec x.
\]
\end{theorem}

\begin{proof}
Supponiamo che $f$ sia una funzione R-integrabile, 
limitata con supporto limitato.
Allora posto 
\[
    g(\vec x) = \int f(\vec x, \vec y)\, d\vec y\\
\]
si ha
\begin{align*}
    S^N(g)
    &= 2^{-nN} \sum_{\vec k\in \ZZ^n} 
        \sup_{\vec x\in Q^N_{\vec k}}
            \int f(\vec x,\vec y)\, dy\\
    &\le 2^{-nN} \sum_{\vec k\in \ZZ^n} 
        \sup_{\vec x\in Q^N_{\vec k}}
                S^N(f(\vec x,\cdot))\\
    &\le 2^{-nN} \sum_{\vec k\in \ZZ^n} 
        \sup_{\vec x\in Q^N_{\vec k}}
            2^{-mN} \sum_{\vec j \in \ZZ^m} 
                \sup_{\vec y \in Q^N_{\vec j}} 
                    f(\vec x, \vec y) \\
    &\le 2^{-(n+m)N} \sum_{\vec k\in \ZZ^n} 
        \sum_{\vec j \in \ZZ^m} 
            \sup_{f(Q^N_{\vec k}\times Q^N_{\vec j})}\\
    &= S^N(f)
\end{align*}
ed analogamente si trova
\[
    S_N(g) \ge S_N(f).
\]
Passando al limite otteniamo quindi 
\[
   S_*(f) \le S_*(g) \le S^*(g) \le S^*(f)
\]
e dunque se $f$ è integrabile (in $n+m$ variabili) anche $g$ è integrabile
(in $n$ variabili) 
e risulta $\int f = \int g$, come volevamo dimostrare.

Gli altri casi si ottengono di conseguenza.
\end{proof}

\begin{theorem}[significato geometrico del determinante]
  \label{th:geometria_determinante}
Sia $L\colon \RR^n\to \RR^n$ una applicazione lineare affine
\[
  L(\vec x) = M\vec x + \vec q.
\]
Allora se $E\subset \RR^n$ è PJ-misurabile si ha 
\[
  \abs{L(E)} = \abs{\det M} \cdot \abs{E}.
\]
\end{theorem}
\begin{proof}
Si vedano appunti di Analisi Uno.
\end{proof}

\begin{lemma}[misurabilità delle immagini Lipschitz]
Sia $E\subset \RR^n$ un compatto misurabile e sia 
$\phi\colon E\to \RR^n$ una funzione $L$-lipschitziana
tale che $\phi$ ristretta alla parte interna di $E$ è iniettiva.
Allora $\phi(E)$ è misurabile e 
\[
  \abs{\phi(E)} \le (L\sqrt n)^n \abs{E}. 
\]
\end{lemma}
\begin{proof}
Infatti essendo $\phi$ continua $\phi(E)$ è compatto e 
visto che $\phi$ è iniettiva sui punti interni di $E$, 
$\phi$ manda punti interni ad $E$ in punti interni a $\phi(E)$.
Dunque $\partial \phi(E) \subset \phi(\partial E)$.

Ora osserviamo che se $Q$ è un cubo di lato $\ell$ la sua immagine $\phi(Q)$ 
è contenuto in un cubo di lato $L \ell \sqrt n$.

Dunque 
\[
 S^N(1_{\partial \phi(E)})\le (L 2^{-N} \sqrt n + 2\cdot 2^{-N})^n
 \to 0 \qquad\text{per $N\to +\infty$}
\]
e dunque $\partial \phi(E)$ ha misura nulla e $\phi(E)$ è misurabile.

Inoltre 
\[
  S^N(1_{\phi(E)}) \le 2^{nN} S^N(1_E) \cdot (L \sqrt n 2^{-N}+2\cdot 2^{-N})^d 
  = S^N(1_E) (L\sqrt n + 2)^n  
\]
e passando al limite per $N\to +\infty$ si ottiene la stima enunciata.
\end{proof}

\begin{theorem}[formula del volume]
  \label{th:formula_volume}
Sia $\Omega\subset \RR^n$ un aperto limitato, $PJ$-misurabile e sia 
$\phi\colon \bar \Omega\to \RR^n$
una funzione di classe $C^1$ che ristretta ad $\Omega$ risulti essere 
iniettiva e con $\det D\phi\neq 0$.

Allora $\phi(\bar \Omega)$ è misurabile e
\begin{equation}\label{eq:area}
 \abs{\phi(\bar \Omega)} = \abs{\phi(\Omega)} = \int_\Omega \abs {\det D\phi}.
\end{equation}

Può essere utile osservare che in queste ipotesi si ha
\begin{equation}\label{eq:49638}
  \overline{\phi(\Omega)} = \phi(\bar \Omega), \qquad
          \partial \phi(\bar \Omega)
  \subset \partial \phi(\Omega)
  = \phi(\partial \Omega).
\end{equation}
\end{theorem}
%
\begin{proof}
\emph{Passo 1: dimostriamo che vale~\eqref{eq:49638}}.
Per continuità di $\phi$ sappiamo che 
$\phi(\bar \Omega) \subset \overline{\phi \Omega}$.
Ma visto che $\bar \Omega$ è compatto, $\phi(\bar \Omega)$ 
è anch'esso compatto e dunque $\phi(\bar \Omega) = \overline{\phi(\Omega)}$.

Se $x\in \Omega$ visto che $\phi$ è localmente invertibile 
si ha che $\phi(x)$ è un punto interno a $\phi(\Omega)$.
Dunque $\phi(\Omega)$ è aperto.
Ne consegue che $\partial \phi(\Omega) = \phi(\partial \Omega)$
e $\partial \phi(\bar \Omega) \subset \partial \phi(\Omega)$.

\emph{Passo 2: $\phi(\Omega)$ è misurabile.}
Essendo $\phi\in C^1(\bar \Omega)$ si intende che $\phi$ è la restrizione 
di una funzione $\C^1$ definita su un aperto $U\supset \bar \Omega$.
Sia $N$ abbastanza grande in modo tale che se $Q^N_{\vec k}$ interseca 
$\Omega$ allora $Q^N_{\vec k}\subset U$. 
Sia $L = \max\ENCLOSE{\Abs{D\phi(\vec x)}\colon \vec x \in Q^N_{\vec k}\subset U }$.
Su ogni $Q^N_{\vec k}$ risulta che $\phi$ è $L$-lipschitz.
Dunque $\phi(Q^N_{\vec k})$ ha diametro non superiore a 
$L \sqrt n 2^{-N}$ e quindi può essere ricoperto con non più di 
$(L\sqrt n + 1)^n$ cubetti di lato $2^{-N}$.
Dunque
\[
  S^N(1_{\partial \phi(\Omega)}) 
  \le (L\sqrt n +1)^n S^N(1_{\partial \Omega})
\]
e per $N\to +\infty$ si ottiene $S^N(1_{\partial \phi(\Omega)})\to 0$
visto che $S^N(1_{\partial \Omega})\to \abs{\partial \Omega} = 0$ 
essendo $\Omega$ PJ-misurabile. 
Dunque anche $\partial\phi(\Omega)$ è misurabile con misura nulla 
e dunque $\phi(\Omega)$ e $\phi(\bar \Omega)$ sono misurabili.

\emph{Passo 3: dimostriamo che se $\bar \Omega=[0,1]^n$ e $\phi$ soddisfa
l'ipotesi
\[
  \abs{\phi(\vec x)-\vec x} \le \eps\abs{x}, \qquad \forall \vec x \in [0,1]^n
\]
per qualche $\eps>0$ allora 
\[
  (1-2\sqrt{n}\eps)^n \le \abs{\phi([0,1]^n)} \le (1+2\sqrt{n}\eps)^n.
\]
}
Infatti si osserva che in queste ipotesi $\abs{\phi(\vec x)-\vec x}\le \sqrt n \eps$
e dunque
\[
  [\sqrt n \eps,1-\sqrt n \eps]^n \subset \phi([0,1]^n) \subset [-\sqrt n \eps,1+\sqrt n \eps]^n
\]
e la tesi segue dalla misurabilità di $\phi([0,1]^n)$ (passo 2)
e dalla monotonia della misura.

\emph{Passo 4: se $\bar \Omega=[0,1]^n$ e $\phi$ soddisfa 
\begin{equation}\label{eq:49924}
  \abs{\phi(\vec x) - M\vec x} \le \eps \abs{\vec x}
\end{equation}
dove $M$ è una matrice invertibile e $\eps>0$, allora 
\[
    \enclose{1-\frac{2\sqrt n \eps}\lambda}^n
    \le \frac{\abs{\phi([0,1]^n)}}{\abs{\det M}} 
    \le \enclose{1+\frac{2\sqrt n \eps}\lambda}^n
\]
dove $\lambda = \Abs{M^{-1}}^{-1}$.
}
Infatti posto $\psi(\vec x) = M^{-1}\phi(\vec x)$
si ha 
\[
  \abs{M(\psi(\vec x)-\vec x)} = \abs{\phi(\vec x) - M\vec x} 
    \le \eps\abs{\vec x}
\]
da cui 
\[
  \abs{\psi(\vec x) - \vec x} \le \frac{\eps}{\lambda} \abs{\vec x}
\]
e rifacendosi al passo precedente si ottiene 
\[
  \enclose{1-\frac{2\sqrt n\eps}{\lambda}}^n
  \le \abs{\psi([0,1]^n)} 
  \le \enclose{1+\frac{2\sqrt n\eps}{\lambda}}^n.
\]
La tesi segue ricordando il teorema~\ref{th:geometria_determinante}
per cui $\abs{\psi([0,1]^n)} = \abs{\det M^{-1}} \cdot \abs{\phi([0,1]^n)}$
e ricordando che $\det M^{-1}= 1/\det(M)$.

\emph{Passo 5. Supponiamo $\bar \Omega=[0,1]^n$
e supponiamo che $\Abs{D\phi(\vec x)^{-1}}^{-1}=\lambda>0$
per ogni $\vec x \in [0,1]^n$.}
Fissiamo $\eps>0$.
Siccome $D\phi$ è una funzione continua definita su un compatto,
è anche uniformemente continua.
Dunque esiste $\delta>0$ tale che 
\[
 \abs{\vec x - \vec y} \le \delta \implies
 \abs{D\phi(\vec x)-D\phi(\vec y)} \le \eps.
\]
Consideriamo $N$ abbastanza grande per cui $\sqrt n 2^{-N} \le \delta$.
Per ogni $\vec k\in \ZZ^n$ tale che $Q^N_{\vec k}\subset [0,1]^n$
posto $\vec x^N_{\vec k} = 2^{-N} \vec k$
e $\Phi^N_{\vec k}=D\phi(\vec x^N_{\vec k})$
per ogni $\vec x \in Q^N_{\vec k}$
per il teorema di Lagrange si ha 
\[
 \phi(\vec x)-\phi(\vec x^N_{\vec k}) 
 = D\phi(\vec \xi)[\vec x-\vec x^N_{\vec k}]
\]
per un qualche $\vec \xi \in Q^N_{\vec k}$ da cui 
\begin{align*}
\abs{\phi(\vec x) - \phi(\vec x^N_{\vec k})-D\phi(\vec x^N_{\vec k})[\vec x-\vec x^N_{\vec k}]}
& = \abs{(D\phi(\vec \xi)-D\phi(\vec x^N_{\vec k}))[\vec x - \vec x^N_{\vec k}]} \\
& \le \eps \abs{\vec x - \vec x^N_{\vec k}}. 
\end{align*}
Applicando il passo precedente alla funzione $\psi^N_{\vec k}(\vec t) = 2^N (\phi(2^{-N}\vec t)-\phi(\vec x^N_{\vec k}))$
si ottiene 
\[
    \enclose{1-\frac{2\sqrt n\eps}{\lambda}}^n
    \le \frac{\abs{\psi^N_{\vec k}([0,1]^n)}}{\abs{\det \phi^N_{\vec k}}} 
    \le \enclose{1+\frac{2\sqrt n\eps}{\lambda}}^n
\]
e osservando che $\abs{\psi^N_{\vec k}([0,1]^n)} = 2^{nN} \abs{\phi(Q^N_{\vec k})}$,
posto $K^N=\ENCLOSE{\vec k\in \ZZ^n\colon Q^N_{\vec k}\subset [0,1]^N}$,
si ottiene
\[
   \enclose{1-\frac{2\sqrt n \eps}{\lambda}} \sum_{\vec k \in K^N} \abs{\det \Phi^N_{\vec k}}
   \le \sum_{\vec k\in K^N} \abs{\phi(Q^N_{\vec k})}
   \le \enclose{1+\frac{2\sqrt n \eps}{\lambda}} \sum_{\vec k\in K^N} \abs{\det \Phi^N_{\vec k}}
\]
da cui, ricordando la definizione di $S^N$ e $S_N$ data nella 
definizione~\ref{def:integrale_riemann} dell'integrale di Riemann,
si ottiene
\[
   \enclose{1-\frac{2\sqrt n \eps}{\lambda}} S_N(\abs{\det D\phi})
   \le \abs{\phi([0,1]^n)}
-   \le \enclose{1+\frac{2\sqrt n \eps}{\lambda}} S^N(\abs{\det D\phi}). 
\]
Facendo tendere $\eps\to 0$ si ha $N\to+\infty$ e quindi,
sapendo che $\abs{\det D\phi}$ è R-integrabile in quanto continua,
\[
  \int_{[0,1]^n} \abs{\det D\phi} = \abs{\phi([0,1]^n)}.
\]

\emph{Passo 6. Supponiamo $\bar \Omega = Q^N_{\vec k}$ e 
$\sup_{\vec x \in Q^N_{\vec k}} \Abs{D\phi(\vec x)^{-1}}^{-1}=\lambda>0$.}
Posto $\vec x^N_{\vec k} = 2^{-N}\vec k$
Possiamo applicare il passo precedente alla funzione 
\[
  \psi(\vec t) = 2^N\phi(\vec x^N_{\vec k}+2^{-N}\vec t)
\]
osservando che $D\psi(\vec t) = D\phi(\vec x^N_{\vec k}+2^{-N}\vec t)$
e $\abs{\psi([0,1]^n)} = 2^{nN}\abs{\phi(Q^N_{\vec k})}$ per ottenere,
come atteso,
\[
   \int_{Q^N_{\vec k}} \abs{\det D\phi} = \abs{\phi(Q^N_{\vec k})}.
\]

\emph{Conclusione.} 
Nel passo 1 abbiamo visto che $\partial \phi(\Omega)$ è misurabile 
e che per ogni $\eps>0$ è possibile trovare un $N\in \NN$ abbastanza grande 
per cui i cubetti $Q^N_{\vec j}$ che coprono $\partial \Omega$ hanno 
area totale così piccola da far sì che $\bigcup_j \phi(Q^N_{\vec j})$ 
abbia misura inferiore a $\eps$.
Fissato questo valore di $N$ consideriamo 
$K^N = \ENCLOSE{\vec k \in ZZ^n\colon Q^N_{\vec k}\subset \Omega}$
e $E^N$ l'unione di tali cubetti.
Per concludere la dimostrazione sarà sufficiente dimostrare che per i 
cubetti $Q^N_{\vec k}$, con $k\in K^N$ si ha 
\[
  \int_{Q^N_{\vec k}} \abs{\det D\phi} = \abs{\phi(Q^N_{\vec k})}.
\]
Visto che $\det D\phi \neq 0$ su $E^N\subset \Omega$ possiamo considerare 
il minimo $\lambda = \min_{\vec x \in E^N} \Abs{D\phi(\vec x)^{-1}}^{-1}$ 
e si ha $\lambda>0$.
Il risultato segue quindi dal passo precedente.
\end{proof}

\begin{theorem}[cambio di variabile]
Sia $\Omega\subset \RR^n$ un aperto limitato, PJ-misurabile e 
sia $\phi\colon \bar \Omega \to \RR^n$
una funzione di classe $C^1$,
che ristretta ad $\Omega$ sia iniettiva ed abbia $\det D\phi\neq 0$. 
Sia  $f\colon \phi(\bar \Omega)\to \RR$ una funzione 
continua. Allora
\[
    \int_\Omega f(\phi(\vec x))\cdot \abs{\det D \phi(\vec x)}\, d \vec x
    =
    \int_{\phi(\Omega)} f(\vec y) \, d \vec y.
\]

La formula si ricorda con la seguente mnemonica:
\[
\begin{cases}
   \vec y = \phi(\vec x)\\
   d {\vec y} = \abs{\det D\phi(\vec x)}\, d\vec x
\end{cases}
\]
\end{theorem}

\begin{proof}
Per la formula di volume, teorema~\ref{th:formula_volume},
sappiamo che $\phi(\Omega)$ è PJ-misurabile. 
Per il teorema~\ref{th:continua_integrabile} 
di integrabilità delle funzioni continue, sappiamo 
che la funzione $f(\vec y)$ è integrabile su $\phi(\Omega)$
e la funzione $f(\phi(\vec x))\cdot \abs{\det D\phi(\vec x)}$
è integrabile su $\Omega$.

Per ogni $\vec k\in \ZZ$ e $N\in \NN$ definiamo
$\vec x^N_{\vec k} = 2^{-N} \vec k$, 
$f^N_{\vec k} = f(\phi(\vec x^N_{\vec k}))$,
$Q^N_{\vec k} = 2^{-N}(\vec k+[0,1]^n)$,
$\Omega^N_{\vec k} = \Omega\cap Q^N_{\vec k}$.
Fissiamo $\eps>0$.
Essendo $f\circ \phi$ continua su $\bar \Omega$ tale funzione 
è anche uniformemente continua. 
Dunque per $N$ sufficientemente grande si ha 
\[
\vec x\in \Omega^N_{\vec k} 
\implies
\abs{f(\phi(\vec x))-f^N_{\vec k}} \le \eps.
\]
Allora per la formula di volume, teorema~\ref{th:formula_volume},
per ogni $\vec k$ si ha 
\begin{align*}
  \int_{\Omega^N_{\vec k}} f^N_{\vec k} \abs{\det D\phi(x)}\, d\vec x 
  &= f^N_{\vec k}\cdot \abs{\phi(\Omega^N_{\vec k})}
  = \int_{\phi(\Omega^N_{\vec k})} f^N_{\vec k} \, d\vec y.
\end{align*}
Allora
\begin{align*}
  \lefteqn{\abs{\int_{\Omega^N_{\vec k}} f(\phi(\vec x))\abs{\det D\phi(\vec x)}\, d\vec x
  - \int_{\phi(\Omega^N_{\vec k})} f(\vec y)\, d\vec y}}\\
  &\le 
  \int_{\Omega^N_{\vec k}} \abs{f(\phi(\vec x))-f^N_{\vec k}} \abs{\det D\phi(\vec x)}\, d\vec x
  + \int_{\phi(\Omega^N_{\vec k})} \abs{f^N_{\vec k} - f(\vec y)}\, d\vec y \\
  & \le 2 \eps \abs{\phi(\Omega^N_{\vec k})}.
\end{align*}
Sommando su $\vec k$ si ottiene 
\[
  \abs{\int_\Omega f(\phi(\vec x))\, d\vec x - \int_{\phi(\Omega)}f(\vec y)\, d\vec y}
  \le 2 \eps \sum_{\vec k} \abs{\phi(\Omega^N_{\vec k})}
  = 2 \eps \abs{\phi(\Omega)}.      
\]
La dimostrazione si conclude facendo tendere $\eps\to 0$.
\end{proof}
