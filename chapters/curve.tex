\section{Curve e lunghezza}

Sia $X$ uno spazio metrico. Una funzione $u\colon [a,b] \to X$
viene chiamata \emph{curva}.

\begin{definition}[lunghezza di una curva]
Diremo che $P$ è una \emph{suddivisione} di $[a,b]$ se
$P$ è un sottoinsieme finito di $[a,b]$ con $a,b\in P$.
Dunque $P = \{ t_0, t_1, \dots, t_N\}$ con $a= t_0 < t_1 < \dots < t_N = b$.
Ad ogni suddivisione $P$ di $[a,b]$ possiamo associare la
lunghezza della \emph{poligonale} $u(t_0), u(t_1), \dots, u(t_N)$ cioè
\[
  \ell_P(u) \defeq \sum_{k=1}^N d(u(t_{k-1}), u(t_k)).
\]
Definiamo la
\emph{lunghezza} della curva $u$ come
\[
    \ell(u) \defeq \sup \ENCLOSE{ \ell_P(u) \colon P \text{ suddivisione di } [a,b] }.
\]
Spesso sarà utile considerare la funzione $u$ ristretta ad un sottointervallo $[\alpha,\beta]\subset [a,b]$.
Useremo la notazione
\[
  \ell_\alpha^\beta(u) \defeq \ell(u \llcorner [\alpha,\beta])
\] 
cosicché $\ell(u) = \ell_a^b(u)$.

In generale $\ell(u)\in [0, +\infty]$.
Se $\ell(u) < +\infty$, diremo che la curva è \emph{rettificabile}
ovvero che ha \emph{lunghezza finita}.
\end{definition}

\begin{theorem}[proprietà della lunghezza]
Sia $\vec u \colon [a,b] \to X$ una curva in uno spazio metrico $(X,d)$.
Allora valgono le seguenti proprietà.
\begin{enumerate}
\item (disuguaglianza triangolare) $\ell(u) \ge d(u(a), u(b))$;
\item (additività) per ogni $c\in [a,b]$ si ha 
  $\ell_a^b(u) = \ell_a^c(u) + \ell_c^b(u)$.
\item (semicontinuità)
se $u_n\colon [a,b] \to X$ converge puntualmente a $u$
(cioè $u_n(t) \to u(t)$ per ogni $t\in[a,b]$)
allora 
\[
  \ell(u) \le \liminf_{n\to \infty} \ell(u_n).
\]
\end{enumerate}
\end{theorem}
\begin{proof}
La prima proprietà (disuguaglianza triangolare) è ovvia 
considerando la suddivisione minima $P = \{ a,b\}$ 
per cui $\ell(u) \ge \ell_P(u) = d(u(a), u(b))$.

Dimostriamo la seconda proprietà (additività).
Data una suddivisione $P$ di $[a,c]$ e una suddivisione $Q$ di $[c,b]$,
la loro unione $P\cup Q$ è una suddivisione di $[a,b]$. Dunque 
\[
    \ell_a^b(u) \ge \ell_{P\cup Q}(u) = \ell_{P}(u) + \ell_{Q}(u)
\] 
da cui, per l'abitrarietà di $P$ e $Q$ si ottiene 
la disuguaglianza $\ell_a^b(u) \ge \ell_a^c(u) + \ell_c^b(u)$.
Viceversa se $P$ è una suddivisione di $[a,b]$ possiamo 
considerare le due suddivisioni $P_1 = P \cap [a,c] \cup \ENCLOSE c$ e
$P_2 = P \cap [c,b] \cup \ENCLOSE c$ 
e ottenere 
\[
\ell_P(u) \le \ell_{P\cup\ENCLOSE c} 
= \ell_{P_1}(u) + \ell_{P_2}(u) 
\le \ell_a^c(u) + \ell_c^b(u).
\]
Dall'arbitrarietà di $P$ si ottiene la disuguaglianza opposta
$\ell_a^b(u) \le \ell_a^c(u) + \ell_c^b(u)$.

Dimostriamo la semicontinuità.
Fissata una qualunque suddivisione $P={t_0,\dots, t_N}$
per ogni $k=0,\dots, N$ si ha $u_n(t_k)\to u(t_k)$ per $n\to +\infty$ e quindi:
\[
  \ell_P(u) =
  \lim_n \ell_P(u_n)
  = \liminf_n \ell_P(u_n)
  \le \liminf_n \ell(u_n).
\]
Dall'arbitrarietà di $P$ si ottiene la tesi.
\end{proof}

\begin{theorem}[formula della lunghezza]
    Sia $\vec u \colon [a,b] \to \RR^n$ una curva di classe $C^1$.
    Allora la curva è rettificabile e la sua lunghezza è data da
    \[
      \ell(\vec u) = \int_a^b \abs{\vec u'(t)} \, dt.
    \]
\end{theorem}
\begin{proof}
Sia $a = t_0 < t_1 < \dots < t_N = b$ una suddivisione $P$ di $[a,b]$.
Osserviamo che 
\[
  \vec u(t_{k+1}) - \vec u(t_k) = \int_{t_k}^{t_{k+1}} \vec u'(t) \, dt
\]
da cui 
\[
 \abs{\vec u(t_{k+1})-\vec u(t_k)} \le \int_{t_k}^{t_{k+1}} \abs{\vec u'(t)} \, dt
\]
e quindi 
\[
  \ell_P(\vec u) \le \int_a^b \abs{\vec u'(t)} \, dt.
\]
Dall'arbitrarietà di $P$ si ottiene la prima disuguaglianza 
$\ell(\vec u) \le \int_a^b \abs{\vec u'}$.

D'altra parte $\vec u'$ è continua in $[a,b]$ dunque è uniformemente continua.
Dato $\eps>0$ esiste $\delta>0$ tale che
\[
  \abs{t-s} < \delta \implies \abs{\vec u'(t) - \vec u'(s)} < \eps.
\]
Visto che $\abs{\vec u'}$ è integrabile (in quanto continua) esiste una suddivisione
$a = t_0 < t_1 < \dots < t_N = b$ con $t_{k+1}-t_k < \delta$ per cui
\[
  \abs{\int_a^b \abs{\vec u'(t)} \, dt - \sum_{k=0}^{N-1} \inf_{t\in [t_k, t_{k+1}]} \abs{\vec u'(t_k)} (t_{k+1}-t_k)} 
  < \eps.
\]
Allora 
\begin{align*}
  \int_a^b \abs{\vec u'(t)} \, dt
    &\le \sum_{k=0}^{N-1} \abs{\vec u'(t_k)} (t_{k+1}-t_k) + \eps \\
    &\le \sum_{k=0}^{N-1} \abs{ \int_{t_k}^{t_{k+1}} \enclose{\vec u'(t_k) - \vec u'(t)} \, dt + \int_{t_k}^{t_{k+1}}\vec u'(t)\, dt} + \eps \\
    &\le \sum_{k=0}^{N-1} \Enclose{ \eps (t_{k+1}-t_k) + \int_{t_k}^{t_{k+1}} \abs{\vec u'(t)} \, dt} + \eps \\
    &=  \eps + \sum_{k=0}^{N-1}  \abs{\vec u(t_{k+1}) - \vec u(t_k)} + \eps \\
    &\le \ell(\vec u) + 2\eps.
\end{align*}
Dall'arbitrarietà di $\eps>0$ si ottiene la disuguaglianza opposta e la tesi.
\end{proof}

\begin{theorem}[riparametrizzazione]
Sia $u \colon [a,b] \to X$ una curva in uno spazio metrico $X$ 
e sia $\phi \colon [A,B] \to [a,b]$ una funzione.
Possiamo considerare la curva $v \colon [A,B] \to X$ definita da
$v=u \circ \phi$, cioè
\[
  v(s) = u(\phi(s)), \qquad s\in [A,B].
\]

\begin{enumerate}
\item
Se $\phi$ è monotona si ha
\[
  \ell(v) \le \ell(u).
\]
\item Se $\phi\colon [A,B] \to [a,b]$ è continua e surgettiva si ha
\[
  \ell(v) \ge \ell(u).
\]
\item se $\phi\colon [A,B] \to [a,b]$ è bigettiva e continua
si ha:
\[
  \ell(u) = \ell(u\circ \phi).
\]
\end{enumerate}
\end{theorem}
\begin{proof}
Dimostriamo il primo punto, dove si suppone che $\phi:[A,B] \to [a,b]$ 
sia monotona. Per fissare le idee supponiamo $\phi$ sia crescente.
Allora presa una qualunque suddivisione $Q$ di $[A,B]$ si ha che 
$P=\phi(Q)\cup \ENCLOSE{a,b}$ è una suddivisione di $[a,b]$.
Se i punti di $Q$ sono $A = t_0 < t_1 < \dots < t_N = B$ i punti di 
$P$ sono $a \le \phi(t_0) \le \phi(t_1) \le \dots \le \phi(t_N) \le b$.
Osserviamo dunque che alcuni punti consecutivi di questo elenco 
potrebbero coincidere
(se $\phi$ non è iniettiva) ma se si ha $\phi(t_k) = \phi(t_{k+1})$
si ha, ovviamente, $d(u(\phi(t_k)), u(\phi(t_{k+1}))) = 0$. 
Dunque risulta comunque 
\begin{align*}
 \ell_Q(v) &= \sum_{k=1}^N d(v(t_{k-1}), v(t_k))
  = \sum_{k=1}^N d(u(\phi(t_{k-1})), u(\phi(t_k)))\\
  &\le d(u(a), u(t_0)) + \sum_{k=1}^N d(u(\phi(t_{k-1})), u(\phi(t_k))) + d(u(t_N), u(b))\\
  &\le \ell_P(u) \le \ell(u).
\end{align*}
Dall'arbitrarietà di $Q$ si ottiene la disuguaglianza $\ell(v) \le \ell(u)$.

Vediamo il caso in cui $\phi$ è continua e surgettiva. 
In questo caso prendiamo una qualunque suddivisione $P$ di $[a,b]$ 
formata dai punti $a = t_0 < t_1 < \dots < t_N = b$.
Vogliamo ora dimostrare che esiste una suddivisione $Q$ di $[A,B]$
formata da punti $A = s_0 < s_1 < \dots < s_N = B$ tali 
che $\phi(s_k) = t_k$ per ogni $k=0,\dots,N$. 
Si procede induttivamente. Prendiamo $s_0=A$ e osserviamo che $\phi(s_0) = t_0$
in quanto per ipotesi $\phi(A) = a$.
Una volta definiti $A = s_0 < s_1 < \dots < s_n < B$ tali che $\phi(s_k) = t_k$ per $k=0,\dots,n$,
andiamo a definire $s_{n+1}$ se $n < N$.
Visto che $\phi(s_n) = t_n < t_{n+1} \le b = \phi(B)$ 
per il teorema dei valori intermedi esiste $s_{n+1} \in (s_n, B]$ tale che
$\phi(s_{n+1}) = t_{n+1}$. 
Procedendo in questo modo si ottiene la suddivisione $Q$ di $[A,B]$ 
con le proprietà richieste e come ultimo punto $s_N$ si può senz'altro scegliere $B$ 
in quanto $\phi(B) = b = t_N$.
Dunque si ha 
\[
  \ell_P(u) = \sum_{k=1}^N d(u(t_{k-1}), u(t_k))
  = \sum_{k=1}^N d(v(s_{k-1}), v(s_k)) = \ell_Q(v) \le \ell(v).
\]
Dall'arbitrarietà di $P$ si ottiene la disuguaglianza voluta $\ell(u) \le \ell(v)$.

Il terzo punto segue immediatamente dai primi due, in quanto una funzione iniettiva e continua 
definita tra due intervalli è necessariamente monotona (per il teorema dei valori intermedi).
\end{proof}

\begin{definition}[lunghezza d'arco]
Sia $u \colon [a,b] \to X$ una curva rettificabile in uno spazio metrico $X$.
Sia $L = \ell(u)$ la lunghezza della curva.
La \emph{lunghezza d'arco} di $u$ è la funzione $\sigma\colon [a,b] \to [0,L]$ 
definita da
\[
  \sigma(t) = \ell_a^t(u).
\]
\end{definition}

Chiaramente la funzione $\sigma$ è crescente ma potrebbe non essere strettamente crescente 
(e quindi non iniettiva) se la curva $u$ ha degli intervalli su cui è costante.

\begin{theorem}[continuità della lunghezza d'arco]
Sia $u \colon [a,b] \to X$ una curva continua rettificabile, di lunghezza $L$, 
in uno spazio metrico $X$.
Allora la funzione lunghezza d'arco $\sigma\colon [a,b] \to [0,L]$ è continua.
\end{theorem}
\begin{proof}
Fissato $t_0\in [a,b)$
mostreremo che $\sigma(t)\to \sigma(t_0)$ per $t\to t_0^+$. Il limite sinistro $t\to t_0^-$ 
quando $t_0\in(a,b]$ si tratta 
in maniera del tutto analoga.

Osserviamo che, per additività della lunghezza, si ha 
$\sigma(t) - \sigma(t_0) = \ell_a^t(u) - \ell_a^{t_0}(u) = \ell_{t_0}^t(u)$.
Dobbiamo dunque dimostrare che $\ell_{t_0}^t(u) \to 0$ per $t\to t_0^+$.

Fissato $\eps>0$ dalla rettificabilità di $u$ 
possiamo trovare dei punti $t_1, t_2, \dots, t_N \in (t_0,b]$
tali che $t_0 < t_1 < t_2 < \dots < t_N$ e
\[
\ell_{t_0}^b(u) 
&\le \sum_{k=1}^N d(u(t_{k-1}), u(t_k)) + \eps
\]
da cui
\begin{align*}
\ell_{t_0}^b(u) 
   &\le d(u(t_0), u(t_1)) + \sum_{k=2}^N d(u(t_{k-1}), u(t_k)) + \eps \\
   &\le d(u(t_0), u(t_1)) + \ell_{t_1}^b(u) + \eps
\end{align*}
ovvero
\[
  \ell_{t_0}^{t_1}(u) = \ell_{t_0}^b(u) - \ell_{t_1}^b(u) \le d(u(t_0), u(t_1)) + \eps.
\]
Scelto ora un qualunque punto $t\in (t_0, t_1)$ si ha
\begin{align*}
  \ell_{t_0}^t(u) 
    &= \ell_{t_0}^{t_1}(u) - \ell_t^{t_1}(u) \\
    &\le d(u(t_0), u(t_1)) + \eps - \ell_t^{t_1}(u) \\
    &\le d(u(t_0), u(t)) + d(u(t), u(t_1)) + \eps - \ell_t^{t_1}(u) \\
    &\le d(u(t_0), u(t)) + \eps.
\end{align*}
Essendo $u$ continua si ha $d(u(t_0), u(t)) \to 0$ per $t \to t_0$ 
e dunque
\[
\limsup_{t\to t_0^+} \ell_{t_0}^t(u) \le \eps.
\]
Ma questo è vero per ogni $\eps>0$, dunque si ottiene, come volevamo dimostrare,
\[
\lim_{t\to t_0^+} \ell_{t_0}^t(u) = 0.
\]
\end{proof}

\begin{definition}[equivalenza di curve]
Due curve $\vec u \colon [a,b] \to X$ e $\vec v \colon [A,B] \to X$ 
si dicono essere \emph{equivalenti} se esiste una funzione
$\phi \colon [A,B] \to [a,b]$ bigettiva e crescente tale che 
\[
    \vec u = \vec v \circ \phi.
\]

Osservazione: se una curva ha dei tratti su cui è costante, in base 
a questa definizione tutte le curve equivalenti avranno anch'esse dei corrispondenti 
tratti costanti. Si potrebbe decidere che per valutare l'equivalenza 
tra due curve si debbano prima eliminare i tratti costanti.
\end{definition}

\begin{theorem}[Ascoli-Arzelà]
Siano $\vec u_n \colon [a,b] \to \RR$ funzioni tali che 
    \begin{enumerate}
    \item le funzioni $\vec u_n$ sono uniformemente limitate, cioè esiste
    $M>0$ tale che per ogni $n$ e per ogni $t\in [a,b]$ si ha 
    $\abs{\vec u_n(t)} \le M$;
    \item le funzioni $\vec u_n$ sono equicontinue, cioè per ogni $\eps>0$ esiste 
    $\delta>0$ tale che per ogni $n$ e per ogni $t_1,t_2 \in [a,b]$ si ha 
    \[
    \abs{t_1-t_2}<\delta \implies 
    \abs{\vec u_n(t_1) - \vec u_n(t_2)} < \eps.
    \]
    \end{enumerate} 
Allora esiste una sottosuccessione $\vec u_{n_k}$ che converge uniformemente
ad una funzione continua $\vec u \colon [a,b] \to \RR$.
\end{theorem}   

\begin{proof}
\emph{Claim: per ogni $\eps>0$ esiste $n_k\in \NN$ successione strettamente 
crescente di indici, 
tale che $\Abs{\vec u_{n_k}(t) - \vec u_{n_j}(t)}_\infty < \eps$ per ogni $k,j\in \NN$.}
Dato $\delta>0$ scegliamo $N\in \NN$ tale che $N \delta < b-a$ e 
consideriamo $a=x_0 < x_1 < x_2 < \dots < x_N=b$ con $x_{i+1}-x_i < \delta$.
La successione $(u_n(x_1), u_n(x_2), \dots, u_n(x_N))\in \RR^N$ 
è limitata dunque, per il teorema di Bolzano-Weierstrass,
ammette una estratta $(u_{n_k}(x_1),\dots,u_{n_k}(x_N))$ convergente in $\RR^N$.
Tale successione è quindi di Cauchy in $\RR^N$. 
Dunque 
dato $\eps>0$ se $k,j$ sono sufficientemente grandi e per ogni $n=1,\dots,N$ si ha
\[
  \abs{u_{n_k}(x_n) - u_{n_j}(x_n)} < \eps.
\]
A meno di eliminare i primi indici della successione possiamo supporre che
questa disuguaglianza valga per ogni $k,j\in \NN$.

Ora, per l'equicontinuità, per ogni $t\in [a,b]$ esiste $x_n$ tale che
$\abs{t - x_n} < \delta$ e dunque
\begin{align*}
  \abs{u_{n_k}(t) - u_{n_j}(t)} 
  &\le \abs{u_{n_k}(t) - u_{n_k}(x_n)} + \abs{u_{n_k}(x_n) - u_{n_j}(x_n)} + \abs{u_{n_j}(x_n) - u_{n_j}(t)} \\
  &\le \eps + \eps + \eps = 3 \eps.
\end{align*}
Questo conclude la dimostrazione del claim.

\emph{Conclusione.}
Trovo $n_k^1$ tale che $\Abs{u_{n_k^1} - u_{n_j^1}}_\infty < 1$ per ogni $k,j$.
Poi trovo $n_k^2$ sottosuccessione estratta di $n_k^1$ tale che
$\Abs{u_{n_k^2} - u_{n_j^2}}_\infty < \frac 1 2$ e, 
procedendo in questo modo per induzione,
\[
  \Abs{u_{n_k^m} - u_{n_j^m}}_\infty < \frac 1 m \qquad \forall k,j \in \NN.
\]
La successione diagonale $u_{n_k^k}$ ha la proprietà che
\[
  \Abs{u_{n_k^k} - u_{n_j^j}} \le \frac 1 k \qquad \forall j\ge k.
\]
Quindi $u_{n_k^k}$ è di Cauchy in $C^0([a,b],\RR)$ rispetto alla distanza 
uniforme e, per la completezza di tale spazio, converge uniformemente 
ad una funzione continua $u \colon [a,b] \to \RR$.
\end{proof}

La stessa dimostrazione permette di dimostrare il seguente risultato 
più astratto.

\begin{theorem}[Ascoli-Arzelà più astratto]
Sia $X$ uno spazio metrico compatto, $Y$ uno spazio metrico. 
Sia $u_n\colon X \to Y$ una successione di funzioni tali che 
\begin{enumerate}
    \item per ogni $x\in X$ la successione $u_n(x)$ è relativamente compatta 
    in $Y$ (cioè ammette una estratta convergente);
    \item le funzioni $u_n$ sono equicontinue.
\end{enumerate}
Allora esiste una sottosuccessione $u_{n_k}$ che converge uniformemente
ad una funzione continua $u\colon X \to Y$.
\end{theorem}

Come applicazione del teorema di Ascoli-Arzelà
proponiamo il seguente importante risultato.

\begin{lemma}[taglio dei lacci]
Sia $u\colon [a,b]\to X$ una curva continua non semplice.
Allora esiste $v\colon 
Allora esistono $\bar t \in [a,b]$, e $\delta\in (0,b-\bar t]$
tali che $u(\bar t) = u(\bar t + \delta)$
e posto $\tilde u\colon [a, b-\delta] \to X$
\[
  \tilde u (t) = \begin{cases}
    u(t) & \text{se } t\in [a,\bar t], \\
    u(t+\delta) & \text{se } t\in [\bar t, b-\delta]
  \end{cases}
\]
la curva $\tilde u$ risulta essere semplice e continua, con gli 
stessi estremi di $u$.
Chiaramente 
\[
    \ell(\tilde u) \le \ell(u,a,\bar t) + \ell(u,\bar t + \delta, b)
    \le \ell(u).
\]
\end{lemma}
\begin{proof}
Basta considerare 
\[
  \bar t = \inf \ENCLOSE{t\in[a,b]\colon \exists s\in(t,b]\, u(t)=u(s)}.
\]
\end{proof}

\begin{theorem}[esistenza delle geodetiche]
    Sia $X$ uno spazio metrico completo, proprio 
    (cioè ogni chiuso limitato è compatto), 
    e sia $G(x,y)$ l'insieme di tutte le curve continue di lunghezza finita 
    $u\colon [0,1] \to X$
    tali che $u(0) = x$ e $u(1) = y$.
    Allora, se $G(x,y)$ non è vuoto, il funzionale \emph{lunghezza}
    $\ell\colon G \to \RR$ ammette un minimo.
\end{theorem}
\begin{proof}
    Sia $L = \inf\ell(G)$ e sia 
    $u_n\in G(x,y)$ una successione minimizzante,
    cioè tale che $\ell(u_n) \to L$.
    Posso supporre che le curve $u_n$ siano semplici (iniettive)
    perché altrimenti posso prendere 
    Posto $L_n = \ell(u_n)$ posso considerare la riparametrizzazione 
    di $u_n$ per lunghezza d'arco. 
    Questo mi dà una curva $\tilde u_n \colon [0,L_n] \to X$
    che è $1$-lipschitziana in quanto 
    $\ell(\tilde u_n, s,t) = t-s$.
    Le curve $v_n \colon [0,1] \to X$ definite da
    \[
        v_n(t) = \tilde u_n (L_n t)
    \]
    saranno invece $L_n$-lipschitziane. 
    Poiché $L_n\to L<+\infty$ esiste $M = \max L_n < +\infty$ 
    e le curve $v_n$ sono tutte $M$-lipschitziane e dunque 
    sono equicontinue. 
    Tutte queste curve sono contenute nella chiusura della 
    palla di centro $x$ e raggio $M$, che è compatta
    perché abbiamo supposto $X$ proprio.
    Dunque, per il teorema di Ascoli-Arzelà, esiste una sottosuccessione 
    $v_{n_k}$ che converge uniformemente ad una curva continua
    $v\colon [0,1] \to X$.
    La curva $v$ è $M$-lipschitziana in quanto limite uniforme di 
    $M$-lipschitziane, dunque è di lunghezza finita e 
    appartiene a $G(x,y)$.
    Infine, per la semicontinuità inferiore della lunghezza
    \[
        \ell(v) \le \liminf_{k\to \infty} \ell(v_{n_k}) = L,
    \]
    dunque $v$ è una curva di lunghezza minima.
\end{proof}

