\chapter{funzioni di più variabili}

Norme equivalenti sullo spazio delle matrici. Lipschitzianità delle funzioni $C^1$.
Le funzioni continue mandano compatti in compatti.
Se $A,B$ sono compatti, se $f\colon A\to B$ e continua e bigettiva allora 
$f^{-1}$ è continua (cioè $f$ è un omeomorfismo).

%Definire $C^1(A)$ come lo spazio delle funzioni su $A$ che possono essere 
%estese di classe $C^1$ ad un aperto contenente $A$.

\begin{theorem}[Lagrange vettoriale]
  \label{th:lagrange}
Sia $\Omega\subset \RR^n$ un aperto convesso, 
sia $\vec f\colon \Omega\to \RR^m$ una funzione differenziabile.
Allora dati $\vec x,\vec y \in \Omega$, $\vec x\neq \vec y$, 
esiste $\vec z \in [\vec x,\vec y]$ (il segmento che congiunge $\vec x$ e $\vec y$)
tale che 
\[
 \abs{\vec f(\vec y)- \vec f(\vec x)}^2
 =  D\vec f(\vec z)[\vec y-\vec x]\cdot (\vec f(\vec y)-\vec f(\vec x)).
\]
In particolare si ha 
\[
\abs{\vec f(\vec y)- \vec f(\vec x)}
\le \max_{\vec z\in [\vec x,\vec y]} \abs{D\vec f(\vec z)} \cdot \abs{\vec y-\vec x}.
\]
\end{theorem}

\begin{exercise}[punto di frontiera]
  \label{ex:punto-di-frontiera}
Sia $\gamma\colon [0,1] \to X$ una curva continua in uno 
spazio topologico $X$ e sia $A\subset X$ tale che 
$\gamma(0)\in A$ e $\gamma(1)\notin A$.
Dimostrare che esiste $t \in [0,1]$ tale che $\gamma(t) \in \partial A$.
\end{exercise}



\begin{theorem}[Teorema spettrale]
Sia $V$ uno spazio euclideo di dimensione $n$ e sia 
$A\colon V\times V \to \RR$ 
una forma bilineare simmetrica.
Allora esiste una base ortonormale $v_1,\dots,v_n$ di $V$
tale che $L(v_j,v_k) = 0$ se $j\neq k$. 

Equivalentemente, se $A$ è una matrice reale simmetrica $n\times n$,
allora esiste una matrice ortogonale $P$ (cioè $P^t=P^{-1}$) tale che
$P^t A P$ è diagonale.
\end{theorem}
%
\begin{proof}
La corrispondenza tra i due enunciati si ottiene 
considerando la matrice $P$ le cui colonne sono i vettori
$v_1,\dots,v_n$.

Faremo la dimostrazione per induzione sulla dimensione $n$ di $V$.
Se $n=0$ il risultato è banale. 
Supponiamo dunque $n>0$ e supponiamo per induzione che il teorema sia vero
per spazi di dimensione $n-1$.

Senza perdere di generalità possiamo supporre che sia $V = \RR^n$.
Consideriamo la funzione $f(x) = x^t A x$, $f\colon \RR^n\to \RR$ e restringiamo 
la funzione $f$ alla sfera unitaria $S^{n-1} = \{ x \in \RR^n \colon \abs{x} = 1\}$.
Per il teorema di Weiestrass $f$ ha minimo in un punto $v_1$ su $S^{n-1}$ 
e per il teorema dei moltiplicatori di Lagrange
esiste $\lambda \in \RR$ tale che $\nabla f(v_1) = \lambda \nabla g(v_1)$
dove $g(x) = \abs{x}^2$. Ma $\nabla f(x) = 2 A x$ e $\nabla g(x) = 2 x$, quindi si ottiene
$A v_1 = \lambda v_1$.

Sia $V_1 = v_1^\perp$ lo spazio ortogonale a $v_1$ (rispetto al prodotto scalare canonico di $\RR^n$).
Visto che $A$ è simmetrico, se $(w,v_1)=0$ allora 
$(Aw,v_1) = (Av_1,w) = \lambda (v_1,w) = 0$.
Questo significa che $A$ manda $V_1$ in se stesso.
L'ipotesi induttiva garantisce dunque l'esistenza di una base ortonormale 
$v_2,\dots,v_n$ per cui $(Av_j,v_k) = 0$ se $j\neq k$, $j,k\ge 2$.
Ma per $k\ge 2$ si ha anche $(Av_k,v_1) = (Av_1,v_k) = \lambda (v_1,v_k) = 0$.
Dunque la base ortonormale $v_1, v_2,\dots,v_n$ soddisfa la tesi.
\end{proof}


\begin{theorem}[Teorema di decomposizione singolare]
Sia $V$ uno spazio vettoriale euclideo di dimensione $n$ e sia
sia $W$ uno spazio vettoriale euclideo di dimensione $m$.
Sia $L\colon V\to W$ una applicazione lineare.
Allora esistono basi ortonormali $v_1,\dots,v_n$ di $V$ e 
$w_1,\dots,w_m$ di $W$ tali che $(L v_k,w_j) = 0$ se $j\neq k$.

Equivalentemente se $L$ è una matrice reale $m\times n$ esistono 
due matrici ortogonali $U$ di ordine $m$ e $V$ di ordine $n$ tali che
$ U^t L V$ è una matrice diagonale $m\times n$.
\end{theorem}

\begin{proof}
Nel caso $n=0$ o $m=0$ la dimostrazione è banale perché uno dei due spazi 
avrà una base vuota e per l'altro qualunque base ortonormale va bene. 
Supponiamo dunque $n>0$ e $m>0$ e supponiamo per induzione 
che il teorema sia vero per spazi di dimensione $n-1$ e $m-1$.

Consideriamo la funzione $f\colon \RR^n\times \RR^m \to \RR$ definita da
$f(x,y) = (L x, y)$.
Restringiamo $f$ al compatto $K=S^{n-1}\times S^{m-1}$
formato dai punti $(x,y)\in\RR^n\times \RR^m$ tali che $\abs{x} = 1$ e $\abs{y} = 1$.
Posto $g_1(x,y) = \frac 1 2 \abs{x}^2$ e $g_2(x,y) = \frac 1 2 \abs{y}^2$,
si ha $K=\{(x,y) \colon g_1(x,y) = \frac 1 2, g_2(x,y) = \frac 1 2\}$.
Per il teorema di Weiestrass $f$ ha minimo in un punto $(v_1,w_1)\in K$ 
e per il teorema dei moltiplicatori di Lagrange, si deve avere
$\nabla f(v_1,w_1) = \lambda \nabla g_1(v_1,w_1) + \mu \nabla g_2(v_1,w_1)$.
Un facile calcolo mostra che $\nabla f(x,y) = (L^t y, L x)$,
mentre $\nabla g_1(x,y) = (x,0)$ e $\nabla g_2(x,y) = (0,y)$. 
Dunque si ottiene
\[
L^t w_1 = \lambda v_1, \qquad L v_1 = \mu w_1.
\]
Sia $V_1 = v_1^\perp \subset \RR^n$ e sia $W_1 = w_1^\perp \subset \RR^m$.
Mostriamo che $L$ manda $V_1$ in $W_1$. 
Infatti se $x\in V_1$ allora $(x,v_1) = 0$ e quindi
\[
(L x, w_1) = (x, L^t w_1) = (x, \lambda v_1) = 0.
\]
Inoltre $w_1$ è ortogonale a $L(V_1)$, in quanto 
se $y = L x$ per qualche $x\in V_1$ allora
\[
(y,w_1) = (L x, w_1) = (x, L^t w_1) = (x, \lambda v_1) = 0.
\]
Per ipotesi induttiva esistono delle basi ortonormali $v_2,\dots, v_n$ di $V_1$ 
e $w_2,\dots,w_m$ di $W_1$ tali che
$(L v_k, w_j) = 0$ se $j\neq k$, $j\ge w$, $k\ge 2$.
Ovviamente $v_1,v_2,\dots, v_n$ è una base ortonormale di $V$ e
$w_1,w_2,\dots,w_m$ è una base ortonormale di $W$.
Per $j\ge 2$ si ha 
\[   
  (L v_1, w_j) = \mu (w_1, w_j) = 0 
\]
e per $k\ge 2$ 
\[
  (L v_k, w_1) = (v_k, L^t w_1) = (v_k, \lambda v_1) = 0
\]
dunque la tesi $(L v_k, w_j) = 0$ è verificata per ogni $k\neq j$.
\end{proof}


